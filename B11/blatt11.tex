% Uebungsaufgaben zur Vorlesung MoSi 3

\NeedsTeXFormat{LaTeX2e}
\documentclass[11pt,a4paper]{article}
%%%%%%%%%%%%%%%%%%%%%%%%%%%%%%%%%%%%%%%%%%%%%%%%%%%%%%%%%%%%%%%%%%%%%%%%%%%%%%%%
% Header fuer Uebungszettel                                                    %
% Jochen Siehr                                                                 %
% 2010-10-27                                                                   %
% last-change 2010-12-20                                                       %
%%%%%%%%%%%%%%%%%%%%%%%%%%%%%%%%%%%%%%%%%%%%%%%%%%%%%%%%%%%%%%%%%%%%%%%%%%%%%%%%

\usepackage{german}
\usepackage[T1]{fontenc}
\usepackage[utf8]{inputenc}

\usepackage{geometry}
\usepackage{amsmath,amsthm,amssymb}
\usepackage{epsfig}
\usepackage{setspace}
\usepackage{longtable}
\usepackage{enumerate}
\usepackage{eurosym}
\usepackage{siunitx}
\usepackage{pifont}
\usepackage{color}

\usepackage[ % letztes Paket!
 linkbordercolor={1 1 1},
 citebordercolor={1 1 1},
 menubordercolor={1 1 1},
 filebordercolor={1 1 1},
 urlbordercolor={1 1 1},
 pdfborder={0 0 0},
 pdftitle={Uebungsblatt},
 pdfauthor={Jochen Siehr}
 ]{hyperref}
\usepackage{breakurl} %muss danach kommen: AUSKOMMENTIEREN für pdflatex 

\pagestyle{empty}
\geometry{margin=20mm}
\setlength{\parindent}{0em}


\newcommand{\R}{\mathbbm{R}}
\newcommand{\C}{\mathbbm{C}}
\newcommand{\N}{\mathbbm{N}}
\newcommand{\Z}{\mathbbm{Z}}

\newcommand{\CC}{\mathbb{C}}
\newcommand{\NN}{\mathbb{N}}
\newcommand{\QQ}{\mathbb{Q}}
\newcommand{\RR}{\mathbb{R}}
\newcommand{\ZZ}{\mathbb{Z}}

\newcommand{\CCC}{\mathcal{C}}
\newcommand{\NNN}{\mathcal{N}}
\newcommand{\LLL}{\mathcal{L}}
\newcommand{\OOO}{\mathcal{O}}

\newcommand{\T}{\mathrm{T}}
\newcommand{\eqqcolon}{=\mathrel{\mathop:}}
\newcommand{\coloneqq}{\mathrel{\mathop:}=}

\newcounter{blatt}
\newtheoremstyle{jochen}{}{}{}{}{\bfseries}{}{\newline}{}
\theoremstyle{jochen}
\newtheorem{aufg}{Aufgabe}[blatt]
\newtheorem{loes}{L\"osungsskizze}[blatt]
\newtheorem{ex}{Exercise}[blatt]
\newtheorem{sol}{Solution}[blatt]



\newcommand{\tstep}{\Delta t}
\newcommand{\tend}{t_{\mathrm{end}}}
\newcommand{\bigO}{\mathcal{O}}
\newcommand{\quoterat}{\hspace{\fill}\qed}
\newcommand{\zB}{z.\,B.\ }
\newcommand{\iter}{[\nu]}
\newcommand{\transx}{^{\mathrm{T}}}
\newcommand{\partdiff}[1]{\partial_{#1}}
\newcommand{\trace}{\operatorname{trace}}
\newcommand{\radicand}{\mathcal{R}}
%%%%%%%%%%%%%%%%%%%%%%%%%%%%%%%%%%%%%%%%%%%%%%%%%%%%%%%%%%%%%%%%%%%%%%%%%%%%%%%%


\setcounter{blatt}{11}


\begin{document}
%%%%%%%%%%%%%%%%%%%%%%%%%%%%%%%%%%%%%%%%%%%%%%%%%%%%%%%%%%%%%%%%%%%%%%%%%%%%%%%%
% Kopf fuer Uebungszettel                                                      %
% Jochen Siehr                                                                 %
% 2010-10-27                                                                   %
% last-change 2012-10-11                                                       %
%%%%%%%%%%%%%%%%%%%%%%%%%%%%%%%%%%%%%%%%%%%%%%%%%%%%%%%%%%%%%%%%%%%%%%%%%%%%%%%%

\begin{minipage}{0.49\textwidth}
 \begin{flushleft}
  \href{http://www.uni-ulm.de}{Universit\"at Ulm}\\
  \href{http://www.uni-ulm.de/mawi/mawi-numerik.html}{Institut f\"ur Numerische Mathematik}
 \end{flushleft}
\end{minipage}
\begin{minipage}{0.49\textwidth}
 \begin{flushright}
  \href{http://www.lebiedz.de}{Prof. Dr. Dirk Lebiedz}\\
  \href{http://www.lebiedz.de/gruppe/marcfein/index.html}{Marc Fein}\\
  \href{http://www.siehr.net}{Jochen Siehr}
 \end{flushright}
\end{minipage}
\bigskip
\begin{center}
\textbf{
\"Ubungen \theblatt{} 
zur Modellierung und Simulation III
(WS 2012/13)\\
\url{http://www.uni-ulm.de/mawi/mawi-numerik/lehre/wintersemester-20122013/vorlesung-modellierung-und-simulation-3.html}
}
\end{center}
\bigskip
\hrule
\bigskip

%%%%%%%%%%%%%%%%%%%%%%%%%%%%%%%%%%%%%%%%%%%%%%%%%%%%%%%%%%%%%%%%%%%%%%%%%%%%%%%%

%%%%%%%%%%%%%%%%%%%%%%%%%%%%%%%%%%%%%%%%%%%%%%%%%%%%%%%%%%%%%%%%%%%%%%%%%%%%%%%%

\begin{aufg}[Hopf Bifurkationen: \"Uberg\"ange von Gleichgewichten zu Grenzzyklen]
Betrachte das System
\begin{equation} \label{hopfeasy} 
\begin{aligned}
  \dot{x} &= -y + \mu x + xy^2\\
  \dot{y} &= x + \mu y - x^2
\end{aligned}
\end{equation}
\begin{enumerate}[a)]
\item Zeige, dass eine Hopf-Bifurkation am Fixpunkt $(x^*,y^*) = (0,0)$ mit sich \"anderndem $\mu$ auftritt. F\"ur welches $\mu$ tritt diese ein?
\item Visualisiere die Hopf-Bifurkation numerisch mit den Verfahren aus Blatt 6. 
\item Bestimme die Eigenschaft der Hopf-Bifurkation analytisch: Ist diese super- oder subkritisch? 

\smallskip
\footnotesize{
\textbf{Hinweis: } Schreibe \eqref{hopfeasy} zuerst in der Form \begin{align*} \dot{x} &= -\omega y + f(x,y) \\ \dot{y} &= \omega x + g(x,y), \end{align*} wobei $f$ and $g$ nichtlineare Terme enthalten. Bestimme hierauf \begin{align*} a &= \frac{1}{16}\left(f_{xxx} + f_{xyy} + g_{xxy} + g_{yyy} + \frac{1}{\omega}\{f_{xy}(f_{xx}+f_{yy}) - g_{xy}(g_{xx}+g_{yy}) - f_{xx}g_{xx} + f_{yy}g_{yy}\}\right),\end{align*} wobei die partiellen Ableitungen an $(x^*,y^*)$ ausgewertet werden. Dann gilt f\"ur die Hopf-Bifurkation:\begin{align*}  a &< 0 \quad \text{superkritisch (stabiler Grenzzyklus)} \\ a &> 0 \quad \text{subkritisch (instabiler Grenzzyklus)}.\end{align*} }
\end{enumerate}
\end{aufg}

%------------------------------------------------------------------------------%
\bigskip%\begin{flushright}b.w.\end{flushright}\pagebreak
%------------------------------------------------------------------------------%

\begin{aufg}[Oszillierende chemische Reaktionen]
Ein einfaches dimensionsloses chemisches System stellt der \textbf{Brusselator} dar:
\begin{align*}
\dot{x} &= 1 - (b+1)x + ax^2y \\
\dot{y} &= bx - ax^2y,
\end{align*}
mit $a,b > 0$ und \glqq Konzentrationen\grqq, $x,y \geq 0.$
\begin{enumerate}[a)]
\item Bestimme alle Fixpunkte und klassifiziere sie.
\item Visualisere die Hauptisoklinen und finde die \emph{trapping}-Region des Flusses.
\item Zeige, dass eine Hopf-Bifurkation auftritt und bestimme den kritischen Wert $b = b_{\mathrm{crit}}.$
\item Argumentiere mit dem \emph{Satz von Poincar\'e-Bendixson}, ob es Grenzzyklen f\"ur $b < b_{\mathrm{crit}}$ oder $b > b_{\mathrm{crit}}$ gibt.
\item Was ist die ungef\"ahre Periode des Grenzzyklus f\"ur $b \approx b_{\mathrm{crit}}?$
\item Erstelle numerisch Phasenportraits f\"ur sich \"andernde Parameter (halte beispielsweise $a$ konstant und variiere $b$). Zeichne auch die Hauptisoklinen aus b) ein.
\end{enumerate}
\end{aufg}

\medskip
\hrule

%%%%%%%%%%%%%%%%%%%%%%%%%%%%%%%%%%%%%%%%%%%%%%%%%%%%%%%%%%%%%%%%%%%%%%%%%%%%%%%%

\end{document}
