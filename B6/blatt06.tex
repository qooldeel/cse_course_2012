% Uebungsaufgaben zur Vorlesung MoSi 3
% Blatt 1

\NeedsTeXFormat{LaTeX2e}
\documentclass[11pt,a4paper]{article}
%%%%%%%%%%%%%%%%%%%%%%%%%%%%%%%%%%%%%%%%%%%%%%%%%%%%%%%%%%%%%%%%%%%%%%%%%%%%%%%%
% Header fuer Uebungszettel                                                    %
% Jochen Siehr                                                                 %
% 2010-10-27                                                                   %
% last-change 2010-12-20                                                       %
%%%%%%%%%%%%%%%%%%%%%%%%%%%%%%%%%%%%%%%%%%%%%%%%%%%%%%%%%%%%%%%%%%%%%%%%%%%%%%%%

\usepackage{german}
\usepackage[T1]{fontenc}
\usepackage[utf8]{inputenc}

\usepackage{geometry}
\usepackage{amsmath,amsthm,amssymb}
\usepackage{epsfig}
\usepackage{setspace}
\usepackage{longtable}
\usepackage{enumerate}
\usepackage{eurosym}
\usepackage{siunitx}
\usepackage{pifont}
\usepackage{color}

\usepackage[ % letztes Paket!
 linkbordercolor={1 1 1},
 citebordercolor={1 1 1},
 menubordercolor={1 1 1},
 filebordercolor={1 1 1},
 urlbordercolor={1 1 1},
 pdfborder={0 0 0},
 pdftitle={Uebungsblatt},
 pdfauthor={Jochen Siehr}
 ]{hyperref}
\usepackage{breakurl} %muss danach kommen: AUSKOMMENTIEREN für pdflatex 

\pagestyle{empty}
\geometry{margin=20mm}
\setlength{\parindent}{0em}


\newcommand{\R}{\mathbbm{R}}
\newcommand{\C}{\mathbbm{C}}
\newcommand{\N}{\mathbbm{N}}
\newcommand{\Z}{\mathbbm{Z}}

\newcommand{\CC}{\mathbb{C}}
\newcommand{\NN}{\mathbb{N}}
\newcommand{\QQ}{\mathbb{Q}}
\newcommand{\RR}{\mathbb{R}}
\newcommand{\ZZ}{\mathbb{Z}}

\newcommand{\CCC}{\mathcal{C}}
\newcommand{\NNN}{\mathcal{N}}
\newcommand{\LLL}{\mathcal{L}}
\newcommand{\OOO}{\mathcal{O}}

\newcommand{\T}{\mathrm{T}}
\newcommand{\eqqcolon}{=\mathrel{\mathop:}}
\newcommand{\coloneqq}{\mathrel{\mathop:}=}

\newcounter{blatt}
\newtheoremstyle{jochen}{}{}{}{}{\bfseries}{}{\newline}{}
\theoremstyle{jochen}
\newtheorem{aufg}{Aufgabe}[blatt]
\newtheorem{loes}{L\"osungsskizze}[blatt]
\newtheorem{ex}{Exercise}[blatt]
\newtheorem{sol}{Solution}[blatt]



\newcommand{\tstep}{\Delta t}
\newcommand{\tend}{t_{\mathrm{end}}}
\newcommand{\bigO}{\mathcal{O}}
\newcommand{\quoterat}{\hspace{\fill}\qed}
\newcommand{\zB}{z.\,B.\ }
\newcommand{\iter}{[\nu]}
\newcommand{\transx}{^{\mathrm{T}}}
\newcommand{\partdiff}[1]{\partial_{#1}}
\newcommand{\trace}{\operatorname{trace}}
\newcommand{\radicand}{\mathcal{R}}
%%%%%%%%%%%%%%%%%%%%%%%%%%%%%%%%%%%%%%%%%%%%%%%%%%%%%%%%%%%%%%%%%%%%%%%%%%%%%%%%


\setcounter{blatt}{6}

\begin{document}
%%%%%%%%%%%%%%%%%%%%%%%%%%%%%%%%%%%%%%%%%%%%%%%%%%%%%%%%%%%%%%%%%%%%%%%%%%%%%%%%
% Kopf fuer Uebungszettel                                                      %
% Jochen Siehr                                                                 %
% 2010-10-27                                                                   %
% last-change 2012-10-11                                                       %
%%%%%%%%%%%%%%%%%%%%%%%%%%%%%%%%%%%%%%%%%%%%%%%%%%%%%%%%%%%%%%%%%%%%%%%%%%%%%%%%

\begin{minipage}{0.49\textwidth}
 \begin{flushleft}
  \href{http://www.uni-ulm.de}{Universit\"at Ulm}\\
  \href{http://www.uni-ulm.de/mawi/mawi-numerik.html}{Institut f\"ur Numerische Mathematik}
 \end{flushleft}
\end{minipage}
\begin{minipage}{0.49\textwidth}
 \begin{flushright}
  \href{http://www.lebiedz.de}{Prof. Dr. Dirk Lebiedz}\\
  \href{http://www.lebiedz.de/gruppe/marcfein/index.html}{Marc Fein}\\
  \href{http://www.siehr.net}{Jochen Siehr}
 \end{flushright}
\end{minipage}
\bigskip
\begin{center}
\textbf{
\"Ubungen \theblatt{} 
zur Modellierung und Simulation III
(WS 2012/13)\\
\url{http://www.uni-ulm.de/mawi/mawi-numerik/lehre/wintersemester-20122013/vorlesung-modellierung-und-simulation-3.html}
}
\end{center}
\bigskip
\hrule
\bigskip

%%%%%%%%%%%%%%%%%%%%%%%%%%%%%%%%%%%%%%%%%%%%%%%%%%%%%%%%%%%%%%%%%%%%%%%%%%%%%%%%

%%%%%%%%%%%%%%%%%%%%%%%%%%%%%%%%%%%%%%%%%%%%%%%%%%%%%%%%%%%%%%%%%%%%%%%%%%%%%%%%

\begin{aufg}[Numerische Berechnung von Phasenportraits mehrdimensionaler nichtlinearer Systeme]
Implementiere das \emph{klassische Runge-Kutta} Verfahren 4. Ordnung
\begin{align*} 
 Y_{k+1} &= Y_k + \frac{1}{6}(k_1 + 2(k_2+k_3) + k_4) \\
\intertext{wobei} \\
k_1 &= \tstep \,F(Y_k)\\
k_2 &= \tstep \,F(Y_k + \tfrac{1}{2}k_1) \\
k_3 &= \tstep \,F(Y_k + \tfrac{1}{2}k_2) \\
k_4 &= \tstep \,F(Y_k + k_3)
\end{align*}
f\"ur folgende Modellprobleme und untersuche damit das Verhalten in Abh\"angigkeit von verschiedenen Anfangswerten und Parametern:
\begin{enumerate}[a)]
\item \textbf{Van-der-Pol Oszillator} (\"uberf\"uhrt in ein System erster DGL) \begin{align*} 
  \dot{y}_1 &= y_2 \\
  \dot{y}_2 &= \mu(1- y_1^2)y_2 - y_1
\end{align*}
mit $\mu \in \RR.$ Wie \"andert sich die Eigenschaft des Fixpunktes $(0,0)^{\mathrm{T}}$ f\"ur $-2.5 \leq \mu \leq 2.5$? 
\item \textbf{Lotka-Volterra Beute-R\"auber-Modell} \begin{alignat*}{2} 
  \dot{y}_1 &= y_1(\alpha - \beta y_2) &\qquad\qquad &\text{Beute}\\
  \dot{y}_2 &= -y_2(\gamma - \delta y_1) &\qquad\qquad &\text{R\"auber}
\end{alignat*}
mit $\alpha, \beta, \gamma, \delta > 0.$ 
%\item Lorenz-Attraktor \begin{align*}  %Vielleicht momentan zu schwer
%  \dot{y}_1 &= \sigma(y_2 - y_1) \\
%  \dot{y}_2 &= y_1(\rho- y_3) - y_2 \\
%  \dot{y}_3 &= y_1y_2 - \beta y_3.\end{align*}
\end{enumerate}
(Hierbei wurde die Abh\"angigkeit der Variablen bzgl. der Zeit aus Gr\"unden der Anschaulichkeit fallen gelassen.)\\
Wie sind die Parameter in den jeweiligen Modellen zu verstehen?
\end{aufg}

%------------------------------------------------------------------------------%
\bigskip%\vfill\begin{flushright}b.w.\end{flushright}\pagebreak
%------------------------------------------------------------------------------%

\begin{aufg}[Steife AWP: Explizite vs implizite Verfahren]
\textit{Steife} AWP stellen besondere Herausforderungen an numerische Verfahren.
Ein einfaches steifes AWP ist das \emph{Davis-Skodje} System \begin{align} 
\label{davisskodje}
\dot{y}_1(t) &= -y_1(t) \notag\\
\dot{y}_2(t) &= -\gamma y_2(t) + \frac{(\gamma -1)y_1(t) + \gamma y_1(t)^2}{(1+y_1(t))^3} \qquad \quad \gamma > 0 \notag\\
& \\
y_1(0) &= 3 \notag\\
y_2(0) &= 1.5. \notag
\end{align}
Sei $\gamma = 60$. L\"ose \eqref{davisskodje} auf dem Zeithorizont $[0,1]$ f\"ur die \"aquidistante Schrittweite $\tstep = 0.03$ wie folgt:
\begin{enumerate}[a)]
\item Expliziter Euler $Y_{k+1} = Y_k + \tstep \,F(Y_k), \ k = 0,1,2,\ldots$.
Was stellt man bzgl. der exakten L\"osung \begin{align*} y_1(t) &= c_1 \exp(-t) \\ y_2(t) &= c_2\exp(-\gamma t) + \frac{c_1}{c_1 + \exp(t)}\end{align*} fest? Bestimme hierf\"ur zun\"achst $c_1, c_2$ in Abh\"angigkeit des Anfangswertes. Welches Verhalten sieht man f\"ur $\tstep \rightarrow 0$?
\item \emph{Impliziter Euler} $Y_{k+1} = Y_k + \tstep\, F(Y_{\mathbf{k+1}}), k = 0,1,2,\ldots$. Dies erfordert jedoch das L\"osen des nichtlinearen GS in jedem Zeitschritt: \begin{align*} G(Y_{k+1}) &:= Y_{k+1} - Y_k - \tstep\, F(Y_{k+1}) \stackrel{!}{=} 0\end{align*} welches mit Hilfe des Newton Verfahrens \begin{align*} G'(Y_{k+1}^{\iter})\delta &= -G(Y_{k+1}^{\iter}), \quad Y_{k+1}^{\iter+1} = Y_{k+1}^{\iter} + \delta, \qquad \iter = 0,1,2,\ldots \\ Y_{k+1}^{[0]} &= Y_{k}\end{align*} gel\"ost werden kann. Implementiere den impliziten Euler analog zu Teil a) und benutze dabei die analytische Jacobimatrix. Was f\"allt auf im Vgl. zum expliziten Euler? 
\end{enumerate}
\end{aufg}


\bigskip
\hrule

%%%%%%%%%%%%%%%%%%%%%%%%%%%%%%%%%%%%%%%%%%%%%%%%%%%%%%%%%%%%%%%%%%%%%%%%%%%%%%%%

\end{document}
