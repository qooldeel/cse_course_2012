% Uebungsaufgaben zur Vorlesung MoSi 3
% Blatt 1

\NeedsTeXFormat{LaTeX2e}
\documentclass[11pt,a4paper]{article}
%%%%%%%%%%%%%%%%%%%%%%%%%%%%%%%%%%%%%%%%%%%%%%%%%%%%%%%%%%%%%%%%%%%%%%%%%%%%%%%%
% Header fuer Uebungszettel                                                    %
% Jochen Siehr                                                                 %
% 2010-10-27                                                                   %
% last-change 2010-12-20                                                       %
%%%%%%%%%%%%%%%%%%%%%%%%%%%%%%%%%%%%%%%%%%%%%%%%%%%%%%%%%%%%%%%%%%%%%%%%%%%%%%%%

\usepackage{german}
\usepackage[T1]{fontenc}
\usepackage[utf8]{inputenc}

\usepackage{geometry}
\usepackage{amsmath,amsthm,amssymb}
\usepackage{epsfig}
\usepackage{setspace}
\usepackage{longtable}
\usepackage{enumerate}
\usepackage{eurosym}
\usepackage{siunitx}
\usepackage{pifont}
\usepackage{color}

\usepackage[ % letztes Paket!
 linkbordercolor={1 1 1},
 citebordercolor={1 1 1},
 menubordercolor={1 1 1},
 filebordercolor={1 1 1},
 urlbordercolor={1 1 1},
 pdfborder={0 0 0},
 pdftitle={Uebungsblatt},
 pdfauthor={Jochen Siehr}
 ]{hyperref}
\usepackage{breakurl} %muss danach kommen: AUSKOMMENTIEREN für pdflatex 

\pagestyle{empty}
\geometry{margin=20mm}
\setlength{\parindent}{0em}


\newcommand{\R}{\mathbbm{R}}
\newcommand{\C}{\mathbbm{C}}
\newcommand{\N}{\mathbbm{N}}
\newcommand{\Z}{\mathbbm{Z}}

\newcommand{\CC}{\mathbb{C}}
\newcommand{\NN}{\mathbb{N}}
\newcommand{\QQ}{\mathbb{Q}}
\newcommand{\RR}{\mathbb{R}}
\newcommand{\ZZ}{\mathbb{Z}}

\newcommand{\CCC}{\mathcal{C}}
\newcommand{\NNN}{\mathcal{N}}
\newcommand{\LLL}{\mathcal{L}}
\newcommand{\OOO}{\mathcal{O}}

\newcommand{\T}{\mathrm{T}}
\newcommand{\eqqcolon}{=\mathrel{\mathop:}}
\newcommand{\coloneqq}{\mathrel{\mathop:}=}

\newcounter{blatt}
\newtheoremstyle{jochen}{}{}{}{}{\bfseries}{}{\newline}{}
\theoremstyle{jochen}
\newtheorem{aufg}{Aufgabe}[blatt]
\newtheorem{loes}{L\"osungsskizze}[blatt]
\newtheorem{ex}{Exercise}[blatt]
\newtheorem{sol}{Solution}[blatt]



\newcommand{\tstep}{\Delta t}
\newcommand{\tend}{t_{\mathrm{end}}}
\newcommand{\bigO}{\mathcal{O}}
\newcommand{\quoterat}{\hspace{\fill}\qed}
\newcommand{\zB}{z.\,B.\ }
\newcommand{\iter}{[\nu]}
\newcommand{\transx}{^{\mathrm{T}}}
\newcommand{\partdiff}[1]{\partial_{#1}}
\newcommand{\trace}{\operatorname{trace}}
\newcommand{\radicand}{\mathcal{R}}
%%%%%%%%%%%%%%%%%%%%%%%%%%%%%%%%%%%%%%%%%%%%%%%%%%%%%%%%%%%%%%%%%%%%%%%%%%%%%%%%


\setcounter{blatt}{3}

\begin{document}
%%%%%%%%%%%%%%%%%%%%%%%%%%%%%%%%%%%%%%%%%%%%%%%%%%%%%%%%%%%%%%%%%%%%%%%%%%%%%%%%
% Kopf fuer Uebungszettel                                                      %
% Jochen Siehr                                                                 %
% 2010-10-27                                                                   %
% last-change 2012-10-11                                                       %
%%%%%%%%%%%%%%%%%%%%%%%%%%%%%%%%%%%%%%%%%%%%%%%%%%%%%%%%%%%%%%%%%%%%%%%%%%%%%%%%

\begin{minipage}{0.49\textwidth}
 \begin{flushleft}
  \href{http://www.uni-ulm.de}{Universit\"at Ulm}\\
  \href{http://www.uni-ulm.de/mawi/mawi-numerik.html}{Institut f\"ur Numerische Mathematik}
 \end{flushleft}
\end{minipage}
\begin{minipage}{0.49\textwidth}
 \begin{flushright}
  \href{http://www.lebiedz.de}{Prof. Dr. Dirk Lebiedz}\\
  \href{http://www.lebiedz.de/gruppe/marcfein/index.html}{Marc Fein}\\
  \href{http://www.siehr.net}{Jochen Siehr}
 \end{flushright}
\end{minipage}
\bigskip
\begin{center}
\textbf{
L\"osungen \theblatt{} 
zur Modellierung und Simulation III
(WS 2012/13)\\
\url{http://www.uni-ulm.de/mawi/mawi-numerik/lehre/wintersemester-20122013/vorlesung-modellierung-und-simulation-3.html}
}
\end{center}
\bigskip
\hrule
\bigskip

%%%%%%%%%%%%%%%%%%%%%%%%%%%%%%%%%%%%%%%%%%%%%%%%%%%%%%%%%%%%%%%%%%%%%%%%%%%%%%%%

%%%%%%%%%%%%%%%%%%%%%%%%%%%%%%%%%%%%%%%%%%%%%%%%%%%%%%%%%%%%%%%%%%%%%%%%%%%%%%%%

\begin{loes}[Potentiale]
  Allgemeines Vorgehen:
  \begin{enumerate}
    \item $-\frac{d V}{dx} = f(x)$
    \item $V(x) = \int \frac{d V}{dx}dx = -\int f(x) dx + C$
    \item Setze $C = 0$
    \item Finde Extremalstellen $x^*$, d.h. $V'(x^*) = 0$
    \item Untersuche diese: \begin{itemize} \item[] $V''(x^*) >0$ \ding{220} Minimalstelle \ding{220} stabiler Fixpunkt \item[] $V''(x^*) < 0$ \ding{220} Maximalstelle \ding{220} instabiler Fixpunkt \item Alternativ: Erstelle Schaubild von $V(x) $\end{itemize}
  \end{enumerate}
  \begin{enumerate}[a)]
    \item \begin{align*} -\frac{dV}{dx} &= -x^2 +x \\ V(x) &= \frac{1}{3} x^3 - \frac{1}{2}x^2 + C, \qquad C := 0 \\ V'(x) &= x^2 - x = x(x-1) \stackrel{!}{=} 0 \Rightarrow x_{1,2} = 0,1 \\ V''(x) &= 2x -1 \\ &\Rightarrow V''(0) = -1 \ \textrm{Maximalstelle (instabiler Fixpkt)},  V''(1) = 1 \ \textrm{Minimalstelle (stabiler Fixpunkt)}\end{align*}
\item \begin{align*} V(x) & = \cosh(x) + C, \  V'(x) = \sinh(x) = \frac{e^x - e^{-x}}{2} \stackrel{!}{=}0 \Rightarrow x^* = 0 \\ V''(x) &= \cosh(x) = \frac{e^x + e^{-x}}{2}|_{x^*} = 1 \ \textrm{\ding{220} lok. Mini. (stabiler Fixpunkt)}  \end{align*}
\item \begin{align*} V(x) & = \frac{1}{4}x^4 - \frac{1}{2}x^2 + C, \  V'(x) = x^3 - x = x(x^2-1)\stackrel{!}{=}0 \Rightarrow x^*_{1,2,3} = 0,1,-1 \\ V''(x) &= 3x^2-1 \Rightarrow V''(0) = -1 \ \textrm{Maxi. (instabil)}, \ V''(1) = V''(-1) = 2 \ \textrm{Mini. (bistabil)} \end{align*}
  \end{enumerate}
\end{loes}

\begin{loes}[Numerisches L\"osen einer gew. DGL]
Teil c): \underline{Diskretisierungsfehler:}
Wir betrachten das AWP \begin{equation}\label{awp} \dot x(t) = f(x(t)), \qquad x(t_0) = x_0\end{equation}
Taylorentwicklung der exakten L\"osung nach $t_0$ und Benutzung von \eqref{awp} zum Vereinfachen f\"uhrt auf:
\begin{align*} 
x(t_1) \equiv x(t_0 + \tstep) &\approx x(t_0) + \dot x(t_0)\tstep + \ddot x(t_0)\frac{\tstep^2}{2} + x^{(3)}(t_0)\frac{\tstep^3}{6} + \bigO(\tstep^4)\\
&= x_0 + f(x_0)\tstep + f_x(x_0)f(x_0)\frac{\tstep^2}{2} + \big(f_{xx}(x_0)(f(x_0))^2 + (f_x(x_0))^2f(x_0)\big)\frac{\tstep^3}{6} \\ &+ \bigO(\tstep^4). 
\end{align*}
\"Ahnlich verfahren wir mit der Iterationsvorschrift f\"ur das Trapezverfahrens:
\begin{align*} 
x_1 &= x_0 + \frac{\tstep}{2}f(x_0) + \frac{\tstep}{2}\underbrace{f(x_0 + \tstep f(x_0))}_{\mathrm{Taylor \ um\ } x_0} \\
&= x_0 + \frac{\tstep}{2}f(x_0) + \frac{\tstep}{2}\big(f(x_0) + \tstep f(x_0)f_x(x_0) + \bigO(\tstep^2)\big)\\
&= x_0 + \tstep f(x_0) + \frac{\tstep^2}{2}f(x_0)f_x(x_0) + \bigO(\tstep^3). 
\end{align*}
Also \begin{align*} |x(t_1) - x_1| \approx \frac{1}{6}\big(f_{xx}(x_0)(f(x_0))^2 + (f_x(x_0))^2f(x_0)\big)\tstep^3 \in \bigO(\tstep^3).  \end{align*}
Da wir $N = \tend/\tstep \in \bigO(\tstep^{-1})$ \"aquidistante Schritte voranschreiten, folgt $|x(t_N) - x_N| \in \bigO(\tstep^2).$ \quoterat
\end{loes}


\bigskip
\hrule
\begin{flushright}
\end{flushright}

%%%%%%%%%%%%%%%%%%%%%%%%%%%%%%%%%%%%%%%%%%%%%%%%%%%%%%%%%%%%%%%%%%%%%%%%%%%%%%%%

\end{document}
