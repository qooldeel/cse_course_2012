% Uebungsaufgaben zur Vorlesung MoSi 3
% Blatt 1

\NeedsTeXFormat{LaTeX2e}
\documentclass[11pt,a4paper]{article}
%%%%%%%%%%%%%%%%%%%%%%%%%%%%%%%%%%%%%%%%%%%%%%%%%%%%%%%%%%%%%%%%%%%%%%%%%%%%%%%%
% Header fuer Uebungszettel                                                    %
% Jochen Siehr                                                                 %
% 2010-10-27                                                                   %
% last-change 2010-12-20                                                       %
%%%%%%%%%%%%%%%%%%%%%%%%%%%%%%%%%%%%%%%%%%%%%%%%%%%%%%%%%%%%%%%%%%%%%%%%%%%%%%%%

\usepackage{german}
\usepackage[T1]{fontenc}
\usepackage[utf8]{inputenc}

\usepackage{geometry}
\usepackage{amsmath,amsthm,amssymb}
\usepackage{epsfig}
\usepackage{setspace}
\usepackage{longtable}
\usepackage{enumerate}
\usepackage{eurosym}
\usepackage{siunitx}
\usepackage{pifont}
\usepackage{color}

\usepackage[ % letztes Paket!
 linkbordercolor={1 1 1},
 citebordercolor={1 1 1},
 menubordercolor={1 1 1},
 filebordercolor={1 1 1},
 urlbordercolor={1 1 1},
 pdfborder={0 0 0},
 pdftitle={Uebungsblatt},
 pdfauthor={Jochen Siehr}
 ]{hyperref}
\usepackage{breakurl} %muss danach kommen: AUSKOMMENTIEREN für pdflatex 

\pagestyle{empty}
\geometry{margin=20mm}
\setlength{\parindent}{0em}


\newcommand{\R}{\mathbbm{R}}
\newcommand{\C}{\mathbbm{C}}
\newcommand{\N}{\mathbbm{N}}
\newcommand{\Z}{\mathbbm{Z}}

\newcommand{\CC}{\mathbb{C}}
\newcommand{\NN}{\mathbb{N}}
\newcommand{\QQ}{\mathbb{Q}}
\newcommand{\RR}{\mathbb{R}}
\newcommand{\ZZ}{\mathbb{Z}}

\newcommand{\CCC}{\mathcal{C}}
\newcommand{\NNN}{\mathcal{N}}
\newcommand{\LLL}{\mathcal{L}}
\newcommand{\OOO}{\mathcal{O}}

\newcommand{\T}{\mathrm{T}}
\newcommand{\eqqcolon}{=\mathrel{\mathop:}}
\newcommand{\coloneqq}{\mathrel{\mathop:}=}

\newcounter{blatt}
\newtheoremstyle{jochen}{}{}{}{}{\bfseries}{}{\newline}{}
\theoremstyle{jochen}
\newtheorem{aufg}{Aufgabe}[blatt]
\newtheorem{loes}{L\"osungsskizze}[blatt]
\newtheorem{ex}{Exercise}[blatt]
\newtheorem{sol}{Solution}[blatt]



\newcommand{\tstep}{\Delta t}
\newcommand{\tend}{t_{\mathrm{end}}}
\newcommand{\bigO}{\mathcal{O}}
\newcommand{\quoterat}{\hspace{\fill}\qed}
\newcommand{\zB}{z.\,B.\ }
\newcommand{\iter}{[\nu]}
\newcommand{\transx}{^{\mathrm{T}}}
\newcommand{\partdiff}[1]{\partial_{#1}}
\newcommand{\trace}{\operatorname{trace}}
\newcommand{\radicand}{\mathcal{R}}
%%%%%%%%%%%%%%%%%%%%%%%%%%%%%%%%%%%%%%%%%%%%%%%%%%%%%%%%%%%%%%%%%%%%%%%%%%%%%%%%


\setcounter{blatt}{4 }

\begin{document}
%%%%%%%%%%%%%%%%%%%%%%%%%%%%%%%%%%%%%%%%%%%%%%%%%%%%%%%%%%%%%%%%%%%%%%%%%%%%%%%%
% Kopf fuer Uebungszettel                                                      %
% Jochen Siehr                                                                 %
% 2010-10-27                                                                   %
% last-change 2012-10-11                                                       %
%%%%%%%%%%%%%%%%%%%%%%%%%%%%%%%%%%%%%%%%%%%%%%%%%%%%%%%%%%%%%%%%%%%%%%%%%%%%%%%%

\begin{minipage}{0.49\textwidth}
 \begin{flushleft}
  \href{http://www.uni-ulm.de}{Universit\"at Ulm}\\
  \href{http://www.uni-ulm.de/mawi/mawi-numerik.html}{Institut f\"ur Numerische Mathematik}
 \end{flushleft}
\end{minipage}
\begin{minipage}{0.49\textwidth}
 \begin{flushright}
  \href{http://www.lebiedz.de}{Prof. Dr. Dirk Lebiedz}\\
  \href{http://www.lebiedz.de/gruppe/marcfein/index.html}{Marc Fein}\\
  \href{http://www.siehr.net}{Jochen Siehr}
 \end{flushright}
\end{minipage}
\bigskip
\begin{center}
\textbf{
\"Ubungen \theblatt{} 
zur Modellierung und Simulation III
(WS 2012/13)\\
\url{http://www.uni-ulm.de/mawi/mawi-numerik/lehre/wintersemester-20122013/vorlesung-modellierung-und-simulation-3.html}
}
\end{center}
\bigskip
\hrule
\bigskip

%%%%%%%%%%%%%%%%%%%%%%%%%%%%%%%%%%%%%%%%%%%%%%%%%%%%%%%%%%%%%%%%%%%%%%%%%%%%%%%%

%%%%%%%%%%%%%%%%%%%%%%%%%%%%%%%%%%%%%%%%%%%%%%%%%%%%%%%%%%%%%%%%%%%%%%%%%%%%%%%%


\begin{aufg}[Sattel-Knoten-Bifurkation]\label{aufg:sn-bif}
 Untersuchen Sie die folgenden Differentialgleichungen für verschiedene $r$:
 \begin{enumerate}[(i)]
  \item $\dot{y} = r-y(1-y)$
  \item $\dot{x} = 1 + rx + x^2$
  \item $\dot{z} = r + z - \ln(1+z)$
  \item $\dot{w} = r^2 + w^2$.
 \end{enumerate}
 Zeigen Sie, dass eine Sattel-Knoten-Bifurkation bei einem bestimmten Wert von
 $r$ auftritt, und zeichnen Sie ein Bifurkationsdiagramm.
\end{aufg}

%------------------------------------------------------------------------------%
\bigskip%\vfill\begin{flushright}b.w.\end{flushright}\pagebreak
%------------------------------------------------------------------------------%

\begin{aufg}[Transkritische Bifurkation]
 Untersuchen Sie die folgenden Differentialgleichungen wie in
 Aufgabe~\ref{aufg:sn-bif} mit dem Unterschied, dass hier eine transkritische
 Bifurkation auftritt:
 \begin{enumerate}[(i)]
  \item $\dot{x} = rx + x^2$
  \item $\dot{y} = y - ry(1-y)$.
 \end{enumerate}
\end{aufg}

%------------------------------------------------------------------------------%
\bigskip%\vfill\begin{flushright}b.w.\end{flushright}\pagebreak
%------------------------------------------------------------------------------%

\begin{aufg}[Chemische Kinetik]
 Wir betrachten ein chemisches Reaktionssystem ähnlich dem in Aufgabe~1.1:
 \begin{align*}
  \textrm{A} + \textrm{X} &\stackrel{k_{\pm 1}}{\rightleftharpoons} 2\, \textrm{X} \\
  \textrm{X} + \textrm{B} &\stackrel{k_{2}}{\rightarrow} C.
 \end{align*}
 Wir nehmen an, dass $\textrm{A}$ und $\textrm{B}$ in so großer Konzentration
 vorliegen, dass diese als konstant angesehen werden können.
 \begin{enumerate}[A)]
  \item Leiten Sie mit dem Massenwirkungsgesetz eine Differentialgleichung der
        Form $\dot{x} = c_1 x + c_2 x^2$ für die Konzentration $x$ von
        $\textrm{X}$ her.
  \item Zeigen Sie, dass $x^*=0$ für $k_2 b > k_1 a$ stabil ist (wobei $a$ und $b$
        jeweils die Konzentrationen von $\textrm{A}$ und $\textrm{B}$ sind), und
        erklären Sie, warum dieser Sachverhalt chemisch sinnvoll ist.
  \item Lösen Sie die Differentialgleichung für verschiedene
        Anfangskonzentrationen und Ratenkoeffizienten mit Ihrem Integrator,
        insbesondere im Bereich $k_2 b\approx k_1 a$.
 \end{enumerate}
\end{aufg}


%------------------------------------------------------------------------------%
\bigskip\vfill\begin{flushright}b.w.\end{flushright}\pagebreak
%------------------------------------------------------------------------------%

\begin{aufg}[Laserschwelle]
 Ein Modell für einen Laser ist durch das folgende System gewöhnlicher
 Differentialgleichungen gegeben (nach Milonni und Eberly)
 \begin{align*}
  \dot{n} &= GnN - kn \\
  \dot{N} &= -GnN - fN + p.
 \end{align*}
 Hier ist $n$ die Zahl der Photonen im Laser-Feld, $N$ ist die Anzahl der
 angeregten Photonen. Parameter $G>0$ modelliert die Zunahme für die stimulierte
 Emission, $k$ die Abnahme. Weiter beschreibt $f$ die Abnahme durch spontane
 Emission, $p$ ist die Stärke der optischen Pumpe.
 \begin{enumerate}[A)]
  \item Wir nehmen an, dass die Änderung von $N$ viel schneller relaxiert als
        die Änderung von $n$. Mit der
        \emph{Quasi Steady-State Approximation} (QSSA) nehmen wir an, dass
        $\dot{N} \approx 0$. Leiten sie damit eine eindimensionale
        Differentialgleichung für $n$ her.
  \item Zeigen Sie, dass $n^*=0$ instabil wird für $p>p_c$ mit der
        Laserschwelle $p_c$.
  \item Welchen Typ hat die Bifurkation bei $p_c$?
 \end{enumerate}

 
\end{aufg}


\bigskip
\hrule

%%%%%%%%%%%%%%%%%%%%%%%%%%%%%%%%%%%%%%%%%%%%%%%%%%%%%%%%%%%%%%%%%%%%%%%%%%%%%%%%

\end{document}
