% Uebungsaufgaben zur Vorlesung MoSi 3


\NeedsTeXFormat{LaTeX2e}
\documentclass[11pt,a4paper]{article}
%%%%%%%%%%%%%%%%%%%%%%%%%%%%%%%%%%%%%%%%%%%%%%%%%%%%%%%%%%%%%%%%%%%%%%%%%%%%%%%%
% Header fuer Uebungszettel                                                    %
% Jochen Siehr                                                                 %
% 2010-10-27                                                                   %
% last-change 2010-12-20                                                       %
%%%%%%%%%%%%%%%%%%%%%%%%%%%%%%%%%%%%%%%%%%%%%%%%%%%%%%%%%%%%%%%%%%%%%%%%%%%%%%%%

\usepackage{german}
\usepackage[T1]{fontenc}
\usepackage[utf8]{inputenc}

\usepackage{geometry}
\usepackage{amsmath,amsthm,amssymb}
\usepackage{epsfig}
\usepackage{setspace}
\usepackage{longtable}
\usepackage{enumerate}
\usepackage{eurosym}
\usepackage{siunitx}
\usepackage{pifont}
\usepackage{color}

\usepackage[ % letztes Paket!
 linkbordercolor={1 1 1},
 citebordercolor={1 1 1},
 menubordercolor={1 1 1},
 filebordercolor={1 1 1},
 urlbordercolor={1 1 1},
 pdfborder={0 0 0},
 pdftitle={Uebungsblatt},
 pdfauthor={Jochen Siehr}
 ]{hyperref}
\usepackage{breakurl} %muss danach kommen: AUSKOMMENTIEREN für pdflatex 

\pagestyle{empty}
\geometry{margin=20mm}
\setlength{\parindent}{0em}


\newcommand{\R}{\mathbbm{R}}
\newcommand{\C}{\mathbbm{C}}
\newcommand{\N}{\mathbbm{N}}
\newcommand{\Z}{\mathbbm{Z}}

\newcommand{\CC}{\mathbb{C}}
\newcommand{\NN}{\mathbb{N}}
\newcommand{\QQ}{\mathbb{Q}}
\newcommand{\RR}{\mathbb{R}}
\newcommand{\ZZ}{\mathbb{Z}}

\newcommand{\CCC}{\mathcal{C}}
\newcommand{\NNN}{\mathcal{N}}
\newcommand{\LLL}{\mathcal{L}}
\newcommand{\OOO}{\mathcal{O}}

\newcommand{\T}{\mathrm{T}}
\newcommand{\eqqcolon}{=\mathrel{\mathop:}}
\newcommand{\coloneqq}{\mathrel{\mathop:}=}

\newcounter{blatt}
\newtheoremstyle{jochen}{}{}{}{}{\bfseries}{}{\newline}{}
\theoremstyle{jochen}
\newtheorem{aufg}{Aufgabe}[blatt]
\newtheorem{loes}{L\"osungsskizze}[blatt]
\newtheorem{ex}{Exercise}[blatt]
\newtheorem{sol}{Solution}[blatt]



\newcommand{\tstep}{\Delta t}
\newcommand{\tend}{t_{\mathrm{end}}}
\newcommand{\bigO}{\mathcal{O}}
\newcommand{\quoterat}{\hspace{\fill}\qed}
\newcommand{\zB}{z.\,B.\ }
\newcommand{\iter}{[\nu]}
\newcommand{\transx}{^{\mathrm{T}}}
\newcommand{\partdiff}[1]{\partial_{#1}}
\newcommand{\trace}{\operatorname{trace}}
\newcommand{\radicand}{\mathcal{R}}
%%%%%%%%%%%%%%%%%%%%%%%%%%%%%%%%%%%%%%%%%%%%%%%%%%%%%%%%%%%%%%%%%%%%%%%%%%%%%%%%


\setcounter{blatt}{3}

\begin{document}
%%%%%%%%%%%%%%%%%%%%%%%%%%%%%%%%%%%%%%%%%%%%%%%%%%%%%%%%%%%%%%%%%%%%%%%%%%%%%%%%
% Kopf fuer Uebungszettel                                                      %
% Jochen Siehr                                                                 %
% 2010-10-27                                                                   %
% last-change 2012-10-11                                                       %
%%%%%%%%%%%%%%%%%%%%%%%%%%%%%%%%%%%%%%%%%%%%%%%%%%%%%%%%%%%%%%%%%%%%%%%%%%%%%%%%

\begin{minipage}{0.49\textwidth}
 \begin{flushleft}
  \href{http://www.uni-ulm.de}{Universit\"at Ulm}\\
  \href{http://www.uni-ulm.de/mawi/mawi-numerik.html}{Institut f\"ur Numerische Mathematik}
 \end{flushleft}
\end{minipage}
\begin{minipage}{0.49\textwidth}
 \begin{flushright}
  \href{http://www.lebiedz.de}{Prof. Dr. Dirk Lebiedz}\\
  \href{http://www.lebiedz.de/gruppe/marcfein/index.html}{Marc Fein}\\
  \href{http://www.siehr.net}{Jochen Siehr}
 \end{flushright}
\end{minipage}
\bigskip
\begin{center}
\textbf{
\"Ubungen \theblatt{} 
zur Modellierung und Simulation III
(WS 2012/13)\\
\url{http://www.uni-ulm.de/mawi/mawi-numerik/lehre/wintersemester-20122013/vorlesung-modellierung-und-simulation-3.html}
}
\end{center}
\bigskip
\hrule
\bigskip

%%%%%%%%%%%%%%%%%%%%%%%%%%%%%%%%%%%%%%%%%%%%%%%%%%%%%%%%%%%%%%%%%%%%%%%%%%%%%%%%

%%%%%%%%%%%%%%%%%%%%%%%%%%%%%%%%%%%%%%%%%%%%%%%%%%%%%%%%%%%%%%%%%%%%%%%%%%%%%%%%

\begin{aufg}[Potentiale]
Untersuche die Dynamik folgender Systeme mit dem Potentialansatz: 
  \begin{enumerate}[a)]
  \item $\dot{x} = x(1-x)$
  \item $\dot{x} = -\sinh(x)$
  \item $\dot{x} = -x^3 +x$
 \end{enumerate}
\end{aufg}

%------------------------------------------------------------------------------%
\bigskip%\begin{flushright}b.w.\end{flushright}\pagebreak
%------------------------------------------------------------------------------%

\begin{aufg}[Numerisches L\"osen einer gew\"ohnlichen DGL]
 Eine einfache Methode zum L\"osen eines Anfangswertproblems (AWP)\begin{equation*} \dot{x}(t) = f(x(t)), \qquad x(t_0) = x_0\end{equation*} ist das \textbf{explizite Trapezverfahren}\begin{align*} x_{i+1} &= x_i + \frac{\tstep_i}{2}\left(f(x_i) + f(x_i + \tstep_i f(x_i))\right), \qquad i = 0,1,2,\ldots  \end{align*} 
\begin{enumerate}[a)]
\item Implementiere dieses Verfahren und wende es auf das AWP aus Aufg. 1.2, definiert auf dem Zeithorizont $[0,\tend = 300]$ mit $g = 9.81 \ \si{m.s^{-1}}, k = 0.73 \ \si{kg.m^{-1}}, m = 120 \ \si{kg}$ f\"ur eine \"aquidistante Schrittweite, \zB $\tstep_i = \tstep = 1$, an. 
\item Gib den Fehler bzgl. der exaken L\"osung \begin{equation*} v(t) = \frac{\tanh(t\sqrt{(kg)/m}+1)\sqrt{g}}{\sqrt{k/m}}\end{equation*} aus. 
\item Zeige, dass das Verfahren die globale Fehlerordnung 2 besitzt, d.h. nach $N$ Zeitschritten gilt \begin{equation*} |x(t_N) - x_N| \in \bigO(\tstep^2).\end{equation*}  
Tipp: Taylorentwicklung.
\end{enumerate}
\end{aufg}

%------------------------------------------------------------------------------%
\bigskip%\begin{flushright}b.w.\end{flushright}\pagebreak
%------------------------------------------------------------------------------%
\begin{aufg}[Zusatz: Simple Schrittweitensteuerung]
Eine äquidistante Schrittweite mag nicht immer die beste Wahl sein. Eine einfache \textbf{Schrittweitensteuerung} geht wie folgt: Sei $\hat{x}_i$ die diskrete L\"osung des Trapezverfahrens (2. Ordnung) im $i.$-ten Schritt und $x_i$ die diskr. L\"osung des im Trapezverfahren verwendeten expliziten Eulers (1. Ordnung). Dann errechnet sich der neue Zeitschritt durch \begin{equation} \label{szctrl}\tstep_{i+1} = \tstep_i\cdot\left(\frac{tol}{|\hat{x}_i-x_i|}\right)^{\frac{1}{2}}.\end{equation}   
\end{aufg}
Implementiere \eqref{szctrl} f\"ur das explizite Trapezverfahren und untersuche die Effizienz gegen\"uber einer \"aquidistanten Diskretisierung f\"ur verschiedene Startzeitschritte und Toleranzen $tol$. 

\bigskip
\hrule

%%%%%%%%%%%%%%%%%%%%%%%%%%%%%%%%%%%%%%%%%%%%%%%%%%%%%%%%%%%%%%%%%%%%%%%%%%%%%%%%

\end{document}
