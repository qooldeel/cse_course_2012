% Uebungsaufgaben zur Vorlesung MoSi 3

\NeedsTeXFormat{LaTeX2e}
\documentclass[11pt,a4paper]{article}
%%%%%%%%%%%%%%%%%%%%%%%%%%%%%%%%%%%%%%%%%%%%%%%%%%%%%%%%%%%%%%%%%%%%%%%%%%%%%%%%
% Header fuer Uebungszettel                                                    %
% Jochen Siehr                                                                 %
% 2010-10-27                                                                   %
% last-change 2010-12-20                                                       %
%%%%%%%%%%%%%%%%%%%%%%%%%%%%%%%%%%%%%%%%%%%%%%%%%%%%%%%%%%%%%%%%%%%%%%%%%%%%%%%%

\usepackage{german}
\usepackage[T1]{fontenc}
\usepackage[utf8]{inputenc}

\usepackage{geometry}
\usepackage{amsmath,amsthm,amssymb}
\usepackage{epsfig}
\usepackage{setspace}
\usepackage{longtable}
\usepackage{enumerate}
\usepackage{eurosym}
\usepackage{siunitx}
\usepackage{pifont}
\usepackage{color}

\usepackage[ % letztes Paket!
 linkbordercolor={1 1 1},
 citebordercolor={1 1 1},
 menubordercolor={1 1 1},
 filebordercolor={1 1 1},
 urlbordercolor={1 1 1},
 pdfborder={0 0 0},
 pdftitle={Uebungsblatt},
 pdfauthor={Jochen Siehr}
 ]{hyperref}
\usepackage{breakurl} %muss danach kommen: AUSKOMMENTIEREN für pdflatex 

\pagestyle{empty}
\geometry{margin=20mm}
\setlength{\parindent}{0em}


\newcommand{\R}{\mathbbm{R}}
\newcommand{\C}{\mathbbm{C}}
\newcommand{\N}{\mathbbm{N}}
\newcommand{\Z}{\mathbbm{Z}}

\newcommand{\CC}{\mathbb{C}}
\newcommand{\NN}{\mathbb{N}}
\newcommand{\QQ}{\mathbb{Q}}
\newcommand{\RR}{\mathbb{R}}
\newcommand{\ZZ}{\mathbb{Z}}

\newcommand{\CCC}{\mathcal{C}}
\newcommand{\NNN}{\mathcal{N}}
\newcommand{\LLL}{\mathcal{L}}
\newcommand{\OOO}{\mathcal{O}}

\newcommand{\T}{\mathrm{T}}
\newcommand{\eqqcolon}{=\mathrel{\mathop:}}
\newcommand{\coloneqq}{\mathrel{\mathop:}=}

\newcounter{blatt}
\newtheoremstyle{jochen}{}{}{}{}{\bfseries}{}{\newline}{}
\theoremstyle{jochen}
\newtheorem{aufg}{Aufgabe}[blatt]
\newtheorem{loes}{L\"osungsskizze}[blatt]
\newtheorem{ex}{Exercise}[blatt]
\newtheorem{sol}{Solution}[blatt]



\newcommand{\tstep}{\Delta t}
\newcommand{\tend}{t_{\mathrm{end}}}
\newcommand{\bigO}{\mathcal{O}}
\newcommand{\quoterat}{\hspace{\fill}\qed}
\newcommand{\zB}{z.\,B.\ }
\newcommand{\iter}{[\nu]}
\newcommand{\transx}{^{\mathrm{T}}}
\newcommand{\partdiff}[1]{\partial_{#1}}
\newcommand{\trace}{\operatorname{trace}}
\newcommand{\radicand}{\mathcal{R}}
%%%%%%%%%%%%%%%%%%%%%%%%%%%%%%%%%%%%%%%%%%%%%%%%%%%%%%%%%%%%%%%%%%%%%%%%%%%%%%%%


\setcounter{blatt}{9}


\begin{document}
%%%%%%%%%%%%%%%%%%%%%%%%%%%%%%%%%%%%%%%%%%%%%%%%%%%%%%%%%%%%%%%%%%%%%%%%%%%%%%%%
% Kopf fuer Uebungszettel                                                      %
% Jochen Siehr                                                                 %
% 2010-10-27                                                                   %
% last-change 2012-10-11                                                       %
%%%%%%%%%%%%%%%%%%%%%%%%%%%%%%%%%%%%%%%%%%%%%%%%%%%%%%%%%%%%%%%%%%%%%%%%%%%%%%%%

\begin{minipage}{0.49\textwidth}
 \begin{flushleft}
  \href{http://www.uni-ulm.de}{Universit\"at Ulm}\\
  \href{http://www.uni-ulm.de/mawi/mawi-numerik.html}{Institut f\"ur Numerische Mathematik}
 \end{flushleft}
\end{minipage}
\begin{minipage}{0.49\textwidth}
 \begin{flushright}
  \href{http://www.lebiedz.de}{Prof. Dr. Dirk Lebiedz}\\
  \href{http://www.lebiedz.de/gruppe/marcfein/index.html}{Marc Fein}\\
  \href{http://www.siehr.net}{Jochen Siehr}
 \end{flushright}
\end{minipage}
\bigskip
\begin{center}
\textbf{
\"Ubungen \theblatt{} 
zur Modellierung und Simulation III
(WS 2012/13)\\
\url{http://www.uni-ulm.de/mawi/mawi-numerik/lehre/wintersemester-20122013/vorlesung-modellierung-und-simulation-3.html}
}
\end{center}
\bigskip
\hrule
\bigskip

%%%%%%%%%%%%%%%%%%%%%%%%%%%%%%%%%%%%%%%%%%%%%%%%%%%%%%%%%%%%%%%%%%%%%%%%%%%%%%%%

%%%%%%%%%%%%%%%%%%%%%%%%%%%%%%%%%%%%%%%%%%%%%%%%%%%%%%%%%%%%%%%%%%%%%%%%%%%%%%%%

\begin{aufg}[Kurzversion von Aufgabe 8.2]
 Hier noch einmal das Wesentliche von Aufgabe 8.2, bei der wir stehen geblieben
 waren:

 Die FHN-Gleichungen sind gegeben durch
 \begin{align}
  \dot{u}      =&\ f(u) - v + I_{\rm a}\notag\\
  \dot{v}      =&\ \varepsilon \left(u - \gamma v + \delta\right)\label{eq:FHN}\\
  f(u) \coloneqq&\ u\left(a-u\right)\left(u-1\right)\notag
\end{align}
mit Spannung $u$ und kombinierter Kraft $v$.
Außerdem ist $I_{\rm a}$ eine Stromstärke, die von außen angelegt ist.
Die Parameterwerte sind $\varepsilon = 0.01$, $\gamma = 0.5$,
$\delta = 0$ und $a = -1$.

\begin{enumerate}[(A)]
 \item Nutzen Sie die \texttt{MATLAB}-Codes zu Blatt~7, um (\ref{eq:FHN})
       numerisch zu lösen. Variieren Sie dabei die angelegten Stromstärke
       zwischen 0 und 1.
 \item Plotten Sie ein Phasenportrait des Modells. Zeichnen Sie
       die Nullklinen und die Trajektorie aus (A) ein.
 \item In der Diplomarbeit \url{http://www.siehr.net/publications/Siehr2007.pdf}
       behauptet der Autor auf S.~21, dass bei einem Wert von
       $I_{\rm a}^{\rm H} \coloneqq 0.4763$ eine Hopf-Bifurkation auftritt.
       Überprüfen Sie diese Behauptung mit numerischen Experimenten.
\end{enumerate}
\end{aufg}

%------------------------------------------------------------------------------%
\bigskip%\begin{flushright}b.w.\end{flushright}\pagebreak
%------------------------------------------------------------------------------%

\begin{aufg}[FitzHugh--Nagumo-Modell, Fortsetzung]
Ändern Sie im Modell von Aufgabe 9.1 den Parameter $\delta$ auf $\delta = 0.5$.
Und wählen Sie $I_{\rm a}=0$.
Setzen Sie die Anfangswerte auf die Werte im Gleichgewicht. Variieren Sie die
angelegte Spannung $I_{\rm a}\in [0,0.04]$ und simulieren Sie das System:
In diesem Zustand heißt das System \emph{erregbar}.
\end{aufg}



%------------------------------------------------------------------------------%
\bigskip%\begin{flushright}b.w.\end{flushright}\pagebreak
%------------------------------------------------------------------------------%

\begin{aufg}[Lyapunov-Funktion]
Gegeben sei das System
\begin{equation}\label{eq:2d-lya}
\begin{aligned}
 \dot{x} &= -2x - y^2\\
 \dot{y} &= -x^2 - y.
\end{aligned} 
\end{equation}
Finden Sie eine Lyapunov-Funktion von (\ref{eq:2d-lya}) f\"ur den Fixpunkt
im Ursprung.
\end{aufg}






\bigskip
\hrule

%%%%%%%%%%%%%%%%%%%%%%%%%%%%%%%%%%%%%%%%%%%%%%%%%%%%%%%%%%%%%%%%%%%%%%%%%%%%%%%%

\end{document}
