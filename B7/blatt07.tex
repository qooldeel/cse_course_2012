% Uebungsaufgaben zur Vorlesung MoSi 3
% Blatt 1

\NeedsTeXFormat{LaTeX2e}
\documentclass[11pt,a4paper]{article}
%%%%%%%%%%%%%%%%%%%%%%%%%%%%%%%%%%%%%%%%%%%%%%%%%%%%%%%%%%%%%%%%%%%%%%%%%%%%%%%%
% Header fuer Uebungszettel                                                    %
% Jochen Siehr                                                                 %
% 2010-10-27                                                                   %
% last-change 2010-12-20                                                       %
%%%%%%%%%%%%%%%%%%%%%%%%%%%%%%%%%%%%%%%%%%%%%%%%%%%%%%%%%%%%%%%%%%%%%%%%%%%%%%%%

\usepackage{german}
\usepackage[T1]{fontenc}
\usepackage[utf8]{inputenc}

\usepackage{geometry}
\usepackage{amsmath,amsthm,amssymb}
\usepackage{epsfig}
\usepackage{setspace}
\usepackage{longtable}
\usepackage{enumerate}
\usepackage{eurosym}
\usepackage{siunitx}
\usepackage{pifont}
\usepackage{color}

\usepackage[ % letztes Paket!
 linkbordercolor={1 1 1},
 citebordercolor={1 1 1},
 menubordercolor={1 1 1},
 filebordercolor={1 1 1},
 urlbordercolor={1 1 1},
 pdfborder={0 0 0},
 pdftitle={Uebungsblatt},
 pdfauthor={Jochen Siehr}
 ]{hyperref}
\usepackage{breakurl} %muss danach kommen: AUSKOMMENTIEREN für pdflatex 

\pagestyle{empty}
\geometry{margin=20mm}
\setlength{\parindent}{0em}


\newcommand{\R}{\mathbbm{R}}
\newcommand{\C}{\mathbbm{C}}
\newcommand{\N}{\mathbbm{N}}
\newcommand{\Z}{\mathbbm{Z}}

\newcommand{\CC}{\mathbb{C}}
\newcommand{\NN}{\mathbb{N}}
\newcommand{\QQ}{\mathbb{Q}}
\newcommand{\RR}{\mathbb{R}}
\newcommand{\ZZ}{\mathbb{Z}}

\newcommand{\CCC}{\mathcal{C}}
\newcommand{\NNN}{\mathcal{N}}
\newcommand{\LLL}{\mathcal{L}}
\newcommand{\OOO}{\mathcal{O}}

\newcommand{\T}{\mathrm{T}}
\newcommand{\eqqcolon}{=\mathrel{\mathop:}}
\newcommand{\coloneqq}{\mathrel{\mathop:}=}

\newcounter{blatt}
\newtheoremstyle{jochen}{}{}{}{}{\bfseries}{}{\newline}{}
\theoremstyle{jochen}
\newtheorem{aufg}{Aufgabe}[blatt]
\newtheorem{loes}{L\"osungsskizze}[blatt]
\newtheorem{ex}{Exercise}[blatt]
\newtheorem{sol}{Solution}[blatt]



\newcommand{\tstep}{\Delta t}
\newcommand{\tend}{t_{\mathrm{end}}}
\newcommand{\bigO}{\mathcal{O}}
\newcommand{\quoterat}{\hspace{\fill}\qed}
\newcommand{\zB}{z.\,B.\ }
\newcommand{\iter}{[\nu]}
\newcommand{\transx}{^{\mathrm{T}}}
\newcommand{\partdiff}[1]{\partial_{#1}}
\newcommand{\trace}{\operatorname{trace}}
\newcommand{\radicand}{\mathcal{R}}
%%%%%%%%%%%%%%%%%%%%%%%%%%%%%%%%%%%%%%%%%%%%%%%%%%%%%%%%%%%%%%%%%%%%%%%%%%%%%%%%


\setcounter{blatt}{7}

\begin{document}
%%%%%%%%%%%%%%%%%%%%%%%%%%%%%%%%%%%%%%%%%%%%%%%%%%%%%%%%%%%%%%%%%%%%%%%%%%%%%%%%
% Kopf fuer Uebungszettel                                                      %
% Jochen Siehr                                                                 %
% 2010-10-27                                                                   %
% last-change 2012-10-11                                                       %
%%%%%%%%%%%%%%%%%%%%%%%%%%%%%%%%%%%%%%%%%%%%%%%%%%%%%%%%%%%%%%%%%%%%%%%%%%%%%%%%

\begin{minipage}{0.49\textwidth}
 \begin{flushleft}
  \href{http://www.uni-ulm.de}{Universit\"at Ulm}\\
  \href{http://www.uni-ulm.de/mawi/mawi-numerik.html}{Institut f\"ur Numerische Mathematik}
 \end{flushleft}
\end{minipage}
\begin{minipage}{0.49\textwidth}
 \begin{flushright}
  \href{http://www.lebiedz.de}{Prof. Dr. Dirk Lebiedz}\\
  \href{http://www.lebiedz.de/gruppe/marcfein/index.html}{Marc Fein}\\
  \href{http://www.siehr.net}{Jochen Siehr}
 \end{flushright}
\end{minipage}
\bigskip
\begin{center}
\textbf{
\"Ubungen \theblatt{} 
zur Modellierung und Simulation III
(WS 2012/13)\\
\url{http://www.uni-ulm.de/mawi/mawi-numerik/lehre/wintersemester-20122013/vorlesung-modellierung-und-simulation-3.html}
}
\end{center}
\bigskip
\hrule
\bigskip

%%%%%%%%%%%%%%%%%%%%%%%%%%%%%%%%%%%%%%%%%%%%%%%%%%%%%%%%%%%%%%%%%%%%%%%%%%%%%%%%

%%%%%%%%%%%%%%%%%%%%%%%%%%%%%%%%%%%%%%%%%%%%%%%%%%%%%%%%%%%%%%%%%%%%%%%%%%%%%%%%

\begin{aufg}[Analytische Fixpunktanalyse]
Ermittle analytisch den Fixpunkt des \emph{Van-der-Pol Oszillators} aus
Aufg.\ 6.2 und charakterisiere ihn in Abh\"angigkeit des Parameters
$\mu \in \RR.$ Verifiziere die Aussage mittels der numerischen Phasenportraits
von \"Ubungsblatt 6.
\end{aufg}

%------------------------------------------------------------------------------%
\bigskip%\vfill\begin{flushright}b.w.\end{flushright}\pagebreak
%------------------------------------------------------------------------------%

\begin{aufg}[SIR-Modell nach Kermack und McKendrick, 1927]
Im SIR-Modell für eine Epidemie werden drei Bevölkerungsgruppen unterschieden:
$S>0$ die Gesunden, die sich anstecken können (susceptible), $I>0$ die
Infizierten (infected) und $R>0$ die Personen, welche die Krankheit (lebend)
\"uberwunden haben (recovered), d.\,h.\  andere
nicht mehr anstecken und sich selbst auch nicht mehr anstecken können.
\begin{enumerate}[(A)]
 \item Modellierung: Stellen Sie ein Dgl.-Modell in den Variablen $S$, $I$ und
 $R$ auf, das die Änderung der Variablen über die Zeit beschreibt. Gehen Sie
 davon aus, dass sich die Krankheit ausbreitet, indem sich Gesunde und
 Infizierte begegnen, wobei ein Treffen mit der Infektiosität $\beta>0$ zur
 Ansteckung führt. Infizierte Personen erholen sich mit der konstanten
 Pro-Kopf-Erholungsrate $r>0$. In Analogie zur chemischen Reaktionskinetik kann
 dieser Sachverhalt als
  \begin{equation*}
   S \stackrel{\beta}{\longrightarrow} I \stackrel{r}{\longrightarrow} R
  \end{equation*}
 geschrieben werden. (Tip: Die Dgl.\ f\"ur $I$ lautet $\dot{I}=\beta SI-rI$.)
 \item Im Modell wird eine abgeschlossene Gruppe von Personen betrachtet, und
 es treten keine demographischen Einflüsse wie Geburt oder anderweitiger Tod
 auf. Daher ist die Gesamtzahl der Personen $N:=S+I+R$ konstant. Zeigen Sie
 dies für das Modell. (Damit kann die Dgl.\ f\"ur $R$ entfallen, weil Sie
 den Wert von $R$ zu jeder Zeit $t$ aus den anderen Werten berechnen können.)
 \item Berechnen und klassifizieren Sie alle Gleichgewichte des Systems
 analytisch.
 \item Skizzieren Sie das Vektorfeld und die Nullklinen des Modells.
 \item Angenommen, eine infizierte Person trifft auf eine Gruppe von 499
 gesunden Menschen, die sich nicht impfen lassen und sich auch sonst nicht gegen
 die Krankheit schützen. Wird es eine Epidemie geben? Lösen Sie das Dgl.-Modell
 für verschiedene Parameter, z.\,B. $\beta=0.001$ und $r=0.1$ und
 Populationsgrößen mit den gegebenen \texttt{MATLAB}-Funktionen. Plotten
 Sie das Vektorfeld mit \texttt{quiver} und zeichnen Sie Lösungstrajektorien
 in dieselbe Graphik.
 \item In der Epidemieforschung spielt die Reproduktionsrate $R_0$ eine
 wichtige Rolle: die durchschnittliche Zahl neuer Infizierter, die ein einzelner
 Infizierter in einer komplett anfälligen, gesunden Population erzeugt.
 Ist $R_0>1$, so bricht eine Epidemie aus. Ist $R_0<1$, so wird die Krankheit
 nicht signifikant weiter übertragen.
 Der Fall einer gesunden Population wird durch $S=N$, $R=0$ beschrieben. Eine
 Person ist infiziert. Eine Epidemie tritt im SIR-Modell per Definition genau
 dann auf, wenn $\dot{I}>0$. Stellen Sie
 eine Formel für $R_0$ mit dem kritischen Wert von $\dot{I}$ auf.
 Überprüfen Sie die Formel mit numerischen Tests: Simulationen des Modells für
 verschiedene Werte. (Tip: $R_0=\tfrac{\beta N}{r}$)
 \item Impfung schützt Individuen vor der Ansteckung! Aber auch eine genügend
 hohe Impfrate kann die Bevölkerung vor einer Epidemie schützen. Begründen Sie
 dies mit der Formel für $R_0$.
 \item Nehmen Sie den Einfluss von Impfung in das Modell auf. Gehen Sie dabei
 davon aus, dass die Impfung vor Ausbruch der Krankheit durchgeführt wird, also
 der Anfangswert entsprechend angepasst wird. Führen Sie numerische Simulationen
 durch, die zeigen, dass ab einem bestimmten geimpften Anteil $vN$ der
 Gesamtbevölkerung mit Impfrate $v$ keine Epidemie mehr eintritt.
 Wie hängt $R_0$ von der Impfrate ab?
 \item Bei welchem Wert von $S$ nimmt $I$ sein Maximum an? Berechnen Sie diesen
 Wert analytisch und plotten Sie ihn in Ihr Phasendiagramm.
 Es kann günstig sein, die zwei-di\-men\-sio\-na\-le gewöhnliche
 Differentialgleichung in eine ein-di\-men\-sio\-na\-le zu transformieren:
 \begin{equation*}
  \frac{{\rm d} I}{{\rm d} S} = \frac{\dot{I}}{\dot{S}}=\dots
 \end{equation*}
 Welche Information geht bei der Transformation verloren?
 \item Nehmen wir an, dass die Immunität der erholten Personen nicht dauerhaft
 anhält, so dass sie sich wieder anstecken können. Ist es möglich, dass sich
 eine dauerhafte Epidemie in der Bevölkerung hält?
\end{enumerate}




\end{aufg}


\bigskip
\hrule

%%%%%%%%%%%%%%%%%%%%%%%%%%%%%%%%%%%%%%%%%%%%%%%%%%%%%%%%%%%%%%%%%%%%%%%%%%%%%%%%

\end{document}
