\NeedsTeXFormat{LaTeX2e}
\documentclass[11pt,a4paper]{article}
%%%%%%%%%%%%%%%%%%%%%%%%%%%%%%%%%%%%%%%%%%%%%%%%%%%%%%%%%%%%%%%%%%%%%%%%%%%%%%%%
% Header fuer Uebungszettel                                                    %
% Jochen Siehr                                                                 %
% 2010-10-27                                                                   %
% last-change 2010-12-20                                                       %
%%%%%%%%%%%%%%%%%%%%%%%%%%%%%%%%%%%%%%%%%%%%%%%%%%%%%%%%%%%%%%%%%%%%%%%%%%%%%%%%

\usepackage{german}
\usepackage[T1]{fontenc}
\usepackage[utf8]{inputenc}

\usepackage{geometry}
\usepackage{amsmath,amsthm,amssymb}
\usepackage{epsfig}
\usepackage{setspace}
\usepackage{longtable}
\usepackage{enumerate}
\usepackage{eurosym}
\usepackage{siunitx}
\usepackage{pifont}
\usepackage{color}

\usepackage[ % letztes Paket!
 linkbordercolor={1 1 1},
 citebordercolor={1 1 1},
 menubordercolor={1 1 1},
 filebordercolor={1 1 1},
 urlbordercolor={1 1 1},
 pdfborder={0 0 0},
 pdftitle={Uebungsblatt},
 pdfauthor={Jochen Siehr}
 ]{hyperref}
\usepackage{breakurl} %muss danach kommen: AUSKOMMENTIEREN für pdflatex 

\pagestyle{empty}
\geometry{margin=20mm}
\setlength{\parindent}{0em}


\newcommand{\R}{\mathbbm{R}}
\newcommand{\C}{\mathbbm{C}}
\newcommand{\N}{\mathbbm{N}}
\newcommand{\Z}{\mathbbm{Z}}

\newcommand{\CC}{\mathbb{C}}
\newcommand{\NN}{\mathbb{N}}
\newcommand{\QQ}{\mathbb{Q}}
\newcommand{\RR}{\mathbb{R}}
\newcommand{\ZZ}{\mathbb{Z}}

\newcommand{\CCC}{\mathcal{C}}
\newcommand{\NNN}{\mathcal{N}}
\newcommand{\LLL}{\mathcal{L}}
\newcommand{\OOO}{\mathcal{O}}

\newcommand{\T}{\mathrm{T}}
\newcommand{\eqqcolon}{=\mathrel{\mathop:}}
\newcommand{\coloneqq}{\mathrel{\mathop:}=}

\newcounter{blatt}
\newtheoremstyle{jochen}{}{}{}{}{\bfseries}{}{\newline}{}
\theoremstyle{jochen}
\newtheorem{aufg}{Aufgabe}[blatt]
\newtheorem{loes}{L\"osungsskizze}[blatt]
\newtheorem{ex}{Exercise}[blatt]
\newtheorem{sol}{Solution}[blatt]



\newcommand{\tstep}{\Delta t}
\newcommand{\tend}{t_{\mathrm{end}}}
\newcommand{\bigO}{\mathcal{O}}
\newcommand{\quoterat}{\hspace{\fill}\qed}
\newcommand{\zB}{z.\,B.\ }
\newcommand{\iter}{[\nu]}
\newcommand{\transx}{^{\mathrm{T}}}
\newcommand{\partdiff}[1]{\partial_{#1}}
\newcommand{\trace}{\operatorname{trace}}
\newcommand{\radicand}{\mathcal{R}}
%%%%%%%%%%%%%%%%%%%%%%%%%%%%%%%%%%%%%%%%%%%%%%%%%%%%%%%%%%%%%%%%%%%%%%%%%%%%%%%%

\setcounter{blatt}{7}

\begin{document}
%%%%%%%%%%%%%%%%%%%%%%%%%%%%%%%%%%%%%%%%%%%%%%%%%%%%%%%%%%%%%%%%%%%%%%%%%%%%%%%%
% Kopf fuer Uebungszettel                                                      %
% Jochen Siehr                                                                 %
% 2010-10-27                                                                   %
% last-change 2012-10-11                                                       %
%%%%%%%%%%%%%%%%%%%%%%%%%%%%%%%%%%%%%%%%%%%%%%%%%%%%%%%%%%%%%%%%%%%%%%%%%%%%%%%%

\begin{minipage}{0.49\textwidth}
 \begin{flushleft}
  \href{http://www.uni-ulm.de}{Universit\"at Ulm}\\
  \href{http://www.uni-ulm.de/mawi/mawi-numerik.html}{Institut f\"ur Numerische Mathematik}
 \end{flushleft}
\end{minipage}
\begin{minipage}{0.49\textwidth}
 \begin{flushright}
  \href{http://www.lebiedz.de}{Prof. Dr. Dirk Lebiedz}\\
  \href{http://www.lebiedz.de/gruppe/marcfein/index.html}{Marc Fein}\\
  \href{http://www.siehr.net}{Jochen Siehr}
 \end{flushright}
\end{minipage}
\bigskip
\begin{center}
\textbf{
\"Ubungen \theblatt{} 
zur Modellierung und Simulation III
(WS 2012/13)\\
\url{http://www.uni-ulm.de/mawi/mawi-numerik/lehre/wintersemester-20122013/vorlesung-modellierung-und-simulation-3.html}
}
\end{center}
\bigskip
\hrule
\bigskip

%%%%%%%%%%%%%%%%%%%%%%%%%%%%%%%%%%%%%%%%%%%%%%%%%%%%%%%%%%%%%%%%%%%%%%%%%%%%%%%%

%%%%%%%%%%%%%%%%%%%%%%%%%%%%%%%%%%%%%%%%%%%%%%%%%%%%%%%%%%%%%%%%%%%%%%%%%%%%%%%%


\begin{aufg}[Analytische Fixpunktanalyse]
Wir betrachten nun 2-dimensionale nichtlineare Systeme $\dot{y} = F(y), $ wobei \begin{alignat*}{2} y &:= (y_1,y_2)\transx \in \RR^2 \quad &\text{und}\quad F(y)&:= (F_1(y), F_2(y))\transx. \end{alignat*}
Das Van-der-Pol System ist gegeben durch:
 \begin{align*} 
  \dot{y}_1 &= y_2 \\
  \dot{y}_2 &= \mu(1- y_1^2)y_2 - y_1.
\end{align*}
\begin{enumerate}
\item \textbf{Fixpunktbestimmung:}\\ Bei einem Fixpunkt ist die rechte Seite Null, d.\,h.
\begin{align} F(y_1,y_2) \label{fixed}&= \begin{bmatrix} y_2 \\  \mu(1- y_1^2)y_2 - y_1\end{bmatrix} =  \begin{bmatrix} 0 \\ 0\end{bmatrix}. \end{align}  
Wir l\"osen beide Gleichungen in \eqref{fixed} gleichzeitig: Aus der oberen folgt $y_2 = 0$, eingesetzt in die unter folgt $y_1 = 0.$ Der einzige Fixpunkt ist demnach $y^* = (0,0)\transx.$
\item \textbf{Fixpunktcharakterisierung (Be wise, linearize):} \\ Bilde die Jacobi-Matrix von $F(y)$:
 \begin{align*} F'(y) &= \begin{bmatrix} \partdiff{y_1}F_1 &\partdiff{y_2}F_1 \\\partdiff{y_1}F_2 &\partdiff{y_2}F_2 \end{bmatrix}_{|(y_1,y_2)\transx} = \begin{bmatrix} 0 & 1 \\ -2\mu y_1y_2-1 & \mu - \mu y_1^2\end{bmatrix}.\end{align*} Setzte den Fixpunkt ein und erhalte \begin{align*} F'(y^*) = \underbrace{\begin{bmatrix} 0 & 1 \\ -1 & \mu \end{bmatrix}}_{:= A}.\end{align*}
 Nun berechnen wir die Eigenwerte (EW) von $A$: \begin{align*} \det(A-\lambda I) = \det\begin{bmatrix} -\lambda & 1 \\ -1 & \mu-\lambda\end{bmatrix} = \lambda^2 - \mu \lambda +1 \stackrel{!}{=}0.\end{align*} L\"osen der quadratischen Gleichung f\"uhrt auf \begin{equation*} \lambda_{1,2} = \frac{\mu \pm \sqrt{\mu^2 -4}}{2}.\end{equation*} Folgende F\"alle bzgl. des Parameters $\mu$ ergeben sich f\"ur den \emph{Radikanden} $\radicand := \mu^2 -4$
\begin{itemize}
\item $\radicand > 0 \rightarrow \mu > \pm 2:$ Die EW sind reell und haben unterschiedliche Vorzeichen (\textbf{Sattelpunkt}) oder dasselbe Vorzeichen (\textbf{Knoten}). 
\item $\radicand < 0: $ komplex konjugierte EW (\textbf{Spiralen und Zentren}).
Haben beide EW negative Realteile, dann ist der Punkt \textbf{stabil}. Sind die Eigenwerte rein imagin\"ar, so haben wir \textbf{neutral-stabile Zentren}
\item $\radicand = 0: $ Diese Parabel stellt die Grenze zw. Knoten und spiralen dar. Sterne (Sternknoten) und degenerierte Knoten befinden sich auf jener Parabel und $\mu$ bestimmt die Stabilit\"at der Knoten und Spiralen.
\end{itemize}
Definiere f\"ur eine komplexe Zahl $z = a + bi \in \CC$ mit $\Re(a)$ den \textit{Realteil} und mit $\Im(a)$ den \textit{Imagin\"arteil} von $z$.
D.\,h. also ganz speziell ():
\begin{center}
\begin{tabular}{|l|l|l|}
\hline
$\mu$ & \textbf{Fixpunkt} & \textbf{Eigenwert}\\
\hline \hline
$0 < \mu < 2$ & Instabiler Fokus & $\Re(\lambda_1),\Re(\lambda_2) > 0$\\
$-2 < \mu < 0$ & Stabiler Fokus & $\Re(\lambda_1), \Re(\lambda_2) < 0$ \\
$\mu > 2$ & Instabiler Knoten & $\lambda_1,\lambda_2 > 0$  \\
$\mu < -2$ & Stabiler Knoten & $\lambda_1,\lambda_2 < 0$  \\
$\mu = 2$ & Instabiler Stern & $\Im(\lambda_1),\Im(\lambda_2) = 0, \Re(\lambda_1),\Re(\lambda_2)>0$ \\
$\mu = -2$ & Stabiler Stern & $\Im(\lambda_1),\Im(\lambda_2) = 0, \Re(\lambda_1),\Re(\lambda_2)<0$ \\
$\mu = 0$ & Elliptisch & $\Re(\lambda_1),\Re(\lambda_2) = 0$\\
\hline 
\end{tabular}
\end{center}

\end{enumerate} 
\end{aufg}

%%%%%%%%%%%%%%%%%%%%%%%%%%%%%%%%%%%%%%%%%%%%%%%%%%%%%%%%%%%%%%%%%%%%%%%%%%%%%%%%

\end{document}
