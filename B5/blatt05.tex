% Uebungsaufgaben zur Vorlesung MoSi 3
% Blatt 1

\NeedsTeXFormat{LaTeX2e}
\documentclass[11pt,a4paper]{article}
%%%%%%%%%%%%%%%%%%%%%%%%%%%%%%%%%%%%%%%%%%%%%%%%%%%%%%%%%%%%%%%%%%%%%%%%%%%%%%%%
% Header fuer Uebungszettel                                                    %
% Jochen Siehr                                                                 %
% 2010-10-27                                                                   %
% last-change 2010-12-20                                                       %
%%%%%%%%%%%%%%%%%%%%%%%%%%%%%%%%%%%%%%%%%%%%%%%%%%%%%%%%%%%%%%%%%%%%%%%%%%%%%%%%

\usepackage{german}
\usepackage[T1]{fontenc}
\usepackage[utf8]{inputenc}

\usepackage{geometry}
\usepackage{amsmath,amsthm,amssymb}
\usepackage{epsfig}
\usepackage{setspace}
\usepackage{longtable}
\usepackage{enumerate}
\usepackage{eurosym}
\usepackage{siunitx}
\usepackage{pifont}
\usepackage{color}

\usepackage[ % letztes Paket!
 linkbordercolor={1 1 1},
 citebordercolor={1 1 1},
 menubordercolor={1 1 1},
 filebordercolor={1 1 1},
 urlbordercolor={1 1 1},
 pdfborder={0 0 0},
 pdftitle={Uebungsblatt},
 pdfauthor={Jochen Siehr}
 ]{hyperref}
\usepackage{breakurl} %muss danach kommen: AUSKOMMENTIEREN für pdflatex 

\pagestyle{empty}
\geometry{margin=20mm}
\setlength{\parindent}{0em}


\newcommand{\R}{\mathbbm{R}}
\newcommand{\C}{\mathbbm{C}}
\newcommand{\N}{\mathbbm{N}}
\newcommand{\Z}{\mathbbm{Z}}

\newcommand{\CC}{\mathbb{C}}
\newcommand{\NN}{\mathbb{N}}
\newcommand{\QQ}{\mathbb{Q}}
\newcommand{\RR}{\mathbb{R}}
\newcommand{\ZZ}{\mathbb{Z}}

\newcommand{\CCC}{\mathcal{C}}
\newcommand{\NNN}{\mathcal{N}}
\newcommand{\LLL}{\mathcal{L}}
\newcommand{\OOO}{\mathcal{O}}

\newcommand{\T}{\mathrm{T}}
\newcommand{\eqqcolon}{=\mathrel{\mathop:}}
\newcommand{\coloneqq}{\mathrel{\mathop:}=}

\newcounter{blatt}
\newtheoremstyle{jochen}{}{}{}{}{\bfseries}{}{\newline}{}
\theoremstyle{jochen}
\newtheorem{aufg}{Aufgabe}[blatt]
\newtheorem{loes}{L\"osungsskizze}[blatt]
\newtheorem{ex}{Exercise}[blatt]
\newtheorem{sol}{Solution}[blatt]



\newcommand{\tstep}{\Delta t}
\newcommand{\tend}{t_{\mathrm{end}}}
\newcommand{\bigO}{\mathcal{O}}
\newcommand{\quoterat}{\hspace{\fill}\qed}
\newcommand{\zB}{z.\,B.\ }
\newcommand{\iter}{[\nu]}
\newcommand{\transx}{^{\mathrm{T}}}
\newcommand{\partdiff}[1]{\partial_{#1}}
\newcommand{\trace}{\operatorname{trace}}
\newcommand{\radicand}{\mathcal{R}}
%%%%%%%%%%%%%%%%%%%%%%%%%%%%%%%%%%%%%%%%%%%%%%%%%%%%%%%%%%%%%%%%%%%%%%%%%%%%%%%%


\setcounter{blatt}{5}

\begin{document}
%%%%%%%%%%%%%%%%%%%%%%%%%%%%%%%%%%%%%%%%%%%%%%%%%%%%%%%%%%%%%%%%%%%%%%%%%%%%%%%%
% Kopf fuer Uebungszettel                                                      %
% Jochen Siehr                                                                 %
% 2010-10-27                                                                   %
% last-change 2012-10-11                                                       %
%%%%%%%%%%%%%%%%%%%%%%%%%%%%%%%%%%%%%%%%%%%%%%%%%%%%%%%%%%%%%%%%%%%%%%%%%%%%%%%%

\begin{minipage}{0.49\textwidth}
 \begin{flushleft}
  \href{http://www.uni-ulm.de}{Universit\"at Ulm}\\
  \href{http://www.uni-ulm.de/mawi/mawi-numerik.html}{Institut f\"ur Numerische Mathematik}
 \end{flushleft}
\end{minipage}
\begin{minipage}{0.49\textwidth}
 \begin{flushright}
  \href{http://www.lebiedz.de}{Prof. Dr. Dirk Lebiedz}\\
  \href{http://www.lebiedz.de/gruppe/marcfein/index.html}{Marc Fein}\\
  \href{http://www.siehr.net}{Jochen Siehr}
 \end{flushright}
\end{minipage}
\bigskip
\begin{center}
\textbf{
\"Ubungen \theblatt{} 
zur Modellierung und Simulation III
(WS 2012/13)\\
\url{http://www.uni-ulm.de/mawi/mawi-numerik/lehre/wintersemester-20122013/vorlesung-modellierung-und-simulation-3.html}
}
\end{center}
\bigskip
\hrule
\bigskip

%%%%%%%%%%%%%%%%%%%%%%%%%%%%%%%%%%%%%%%%%%%%%%%%%%%%%%%%%%%%%%%%%%%%%%%%%%%%%%%%

%%%%%%%%%%%%%%%%%%%%%%%%%%%%%%%%%%%%%%%%%%%%%%%%%%%%%%%%%%%%%%%%%%%%%%%%%%%%%%%%


\begin{aufg}[Pitchfork-Bifurkation]
 Analysieren Sie die folgenden Differentialgleichung, und zeigen Sie, dass eine
 Pitchfork-Bifurkation auftreten kann. Ist diese super- oder subkritisch?
 \begin{enumerate}[(i)]
  \item $\dot{x} = x + rx^3$
  \item $\dot{y} = y - ry^3$
  \item $\dot{z} = z + \tfrac{rz}{1+z^2}$.
 \end{enumerate}
\end{aufg}

%------------------------------------------------------------------------------%
\bigskip%\vfill\begin{flushright}b.w.\end{flushright}\pagebreak
%------------------------------------------------------------------------------%

\begin{aufg}[Insektenplage]
Der \emph{spruce budworm} (Fichten-Knospenbohrer?) ist ein Schädling in
Ost-Kanada, der dort ganze Wälder zerstört. Die Dynamik des Waldes kann als
konstant angesehen werden. Die Dynamik der Wurmpopulation kann durch
\begin{equation*}
 \dot{N} = RN \left( 1- \frac{N}{K} \right) - p(N)
\end{equation*}
beschrieben werden.
Die Wurmpopulation wächst logistisch, wenn keine Räuber vorhanden sind. Der
Parameter $K$ wird bestimmt durch die Menge an Laub in den Bäumen und ändert sich
mit den Jahreszeiten.

Der Term $p(N)$ beschreibt den Tod durch \glqq{}gefressen
werden\grqq{}: Ab einer bestimmten Populationsgröße beginnen die Vögel, die
Würmer zu fressen, so schnell sie können. Dieses Verhalten wird durch
\begin{equation*}
 p(N) = \frac{BN^2}{A^2 + N^2}
\end{equation*}
mit $A,B>0$ modelliert.

\begin{enumerate}[A)]
 \item Entdimensionalisieren Sie die Diffentialgleichung, indem Sie sie durch
       $B$ teilen und in die Variable $x:=\tfrac{N}{A}$ transformieren.
       Benutzen Sie auch die dimensionslose Zeit $\tau:=\tfrac{Bt}{A}$ und
       Paramter $r:=\tfrac{RA}{B}$, $k:=\tfrac{K}{A}$.
 \item Bestimmen Sie die Fixpunkte der dimensionslosen Differentialgleichung.
 \item Bestimmen Sie die Stabilität der Fixpunkte (evtl.\ graphisch)
       in Abhängigkeit von $k$ und $r$.
 \item Berechnen Sie Bifurkationskurven: Kurven in der $(k,r)$-Ebene in Abhängigkeit
       von $x$, an deren Punkte eine Bifurkation auftritt.
\end{enumerate}
\end{aufg}

%------------------------------------------------------------------------------%
\bigskip%\vfill\begin{flushright}b.w.\end{flushright}\pagebreak
%------------------------------------------------------------------------------%

\begin{aufg}[Phasenportrait]
Mit dem Befehl \texttt{quiver} können Sie Phasenportraits in \texttt{MATLAB} schnell
und einfach erstellen. Schreiben Sie sich eine Funktion, mit deren Hilfe Sie
die Phasenportraits folgender Systeme ansehen können:
 \begin{enumerate}[(i)]
  \item $\dot{x} = x-y, \quad \dot{y} = 1 - \mathrm{e}^x$
  \item $\dot{x} = x-x^3, \quad \dot{y} = -|y|$
  \item $\dot{x} = y \sin(x), \quad \dot{y} = x^2 - y$.
 \end{enumerate}
\end{aufg}




\bigskip
\hrule

%%%%%%%%%%%%%%%%%%%%%%%%%%%%%%%%%%%%%%%%%%%%%%%%%%%%%%%%%%%%%%%%%%%%%%%%%%%%%%%%

\end{document}
