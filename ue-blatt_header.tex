%%%%%%%%%%%%%%%%%%%%%%%%%%%%%%%%%%%%%%%%%%%%%%%%%%%%%%%%%%%%%%%%%%%%%%%%%%%%%%%%
% Header fuer Uebungszettel                                                    %
% Jochen Siehr                                                                 %
% 2010-10-27                                                                   %
% last-change 2010-12-20                                                       %
%%%%%%%%%%%%%%%%%%%%%%%%%%%%%%%%%%%%%%%%%%%%%%%%%%%%%%%%%%%%%%%%%%%%%%%%%%%%%%%%

\usepackage{german}
\usepackage[T1]{fontenc}
\usepackage[utf8]{inputenc}

\usepackage{geometry}
\usepackage{amsmath,amsthm,amssymb}
\usepackage{epsfig}
\usepackage{setspace}
\usepackage{longtable}
\usepackage{enumerate}
\usepackage{eurosym}
\usepackage{siunitx}
\usepackage{pifont}
\usepackage{color}

\usepackage[ % letztes Paket!
 linkbordercolor={1 1 1},
 citebordercolor={1 1 1},
 menubordercolor={1 1 1},
 filebordercolor={1 1 1},
 urlbordercolor={1 1 1},
 pdfborder={0 0 0},
 pdftitle={Uebungsblatt},
 pdfauthor={Jochen Siehr}
 ]{hyperref}
\usepackage{breakurl} %muss danach kommen: AUSKOMMENTIEREN für pdflatex 

\pagestyle{empty}
\geometry{margin=20mm}
\setlength{\parindent}{0em}


\newcommand{\R}{\mathbbm{R}}
\newcommand{\C}{\mathbbm{C}}
\newcommand{\N}{\mathbbm{N}}
\newcommand{\Z}{\mathbbm{Z}}

\newcommand{\CC}{\mathbb{C}}
\newcommand{\NN}{\mathbb{N}}
\newcommand{\QQ}{\mathbb{Q}}
\newcommand{\RR}{\mathbb{R}}
\newcommand{\ZZ}{\mathbb{Z}}

\newcommand{\CCC}{\mathcal{C}}
\newcommand{\NNN}{\mathcal{N}}
\newcommand{\LLL}{\mathcal{L}}
\newcommand{\OOO}{\mathcal{O}}

\newcommand{\T}{\mathrm{T}}
\newcommand{\eqqcolon}{=\mathrel{\mathop:}}
\newcommand{\coloneqq}{\mathrel{\mathop:}=}

\newcounter{blatt}
\newtheoremstyle{jochen}{}{}{}{}{\bfseries}{}{\newline}{}
\theoremstyle{jochen}
\newtheorem{aufg}{Aufgabe}[blatt]
\newtheorem{loes}{L\"osungsskizze}[blatt]
\newtheorem{ex}{Exercise}[blatt]
\newtheorem{sol}{Solution}[blatt]



\newcommand{\tstep}{\Delta t}
\newcommand{\tend}{t_{\mathrm{end}}}
\newcommand{\bigO}{\mathcal{O}}
\newcommand{\quoterat}{\hspace{\fill}\qed}
\newcommand{\zB}{z.\,B.\ }
\newcommand{\iter}{[\nu]}
\newcommand{\transx}{^{\mathrm{T}}}
\newcommand{\partdiff}[1]{\partial_{#1}}
\newcommand{\trace}{\operatorname{trace}}
\newcommand{\radicand}{\mathcal{R}}
%%%%%%%%%%%%%%%%%%%%%%%%%%%%%%%%%%%%%%%%%%%%%%%%%%%%%%%%%%%%%%%%%%%%%%%%%%%%%%%%
