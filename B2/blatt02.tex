% Uebungsaufgaben zur Vorlesung MoSi 3
% Blatt 1

\NeedsTeXFormat{LaTeX2e}
\documentclass[11pt,a4paper]{article}
%%%%%%%%%%%%%%%%%%%%%%%%%%%%%%%%%%%%%%%%%%%%%%%%%%%%%%%%%%%%%%%%%%%%%%%%%%%%%%%%
% Header fuer Uebungszettel                                                    %
% Jochen Siehr                                                                 %
% 2010-10-27                                                                   %
% last-change 2010-12-20                                                       %
%%%%%%%%%%%%%%%%%%%%%%%%%%%%%%%%%%%%%%%%%%%%%%%%%%%%%%%%%%%%%%%%%%%%%%%%%%%%%%%%

\usepackage{german}
\usepackage[T1]{fontenc}
\usepackage[utf8]{inputenc}

\usepackage{geometry}
\usepackage{amsmath,amsthm,amssymb}
\usepackage{epsfig}
\usepackage{setspace}
\usepackage{longtable}
\usepackage{enumerate}
\usepackage{eurosym}
\usepackage{siunitx}
\usepackage{pifont}
\usepackage{color}

\usepackage[ % letztes Paket!
 linkbordercolor={1 1 1},
 citebordercolor={1 1 1},
 menubordercolor={1 1 1},
 filebordercolor={1 1 1},
 urlbordercolor={1 1 1},
 pdfborder={0 0 0},
 pdftitle={Uebungsblatt},
 pdfauthor={Jochen Siehr}
 ]{hyperref}
\usepackage{breakurl} %muss danach kommen: AUSKOMMENTIEREN für pdflatex 

\pagestyle{empty}
\geometry{margin=20mm}
\setlength{\parindent}{0em}


\newcommand{\R}{\mathbbm{R}}
\newcommand{\C}{\mathbbm{C}}
\newcommand{\N}{\mathbbm{N}}
\newcommand{\Z}{\mathbbm{Z}}

\newcommand{\CC}{\mathbb{C}}
\newcommand{\NN}{\mathbb{N}}
\newcommand{\QQ}{\mathbb{Q}}
\newcommand{\RR}{\mathbb{R}}
\newcommand{\ZZ}{\mathbb{Z}}

\newcommand{\CCC}{\mathcal{C}}
\newcommand{\NNN}{\mathcal{N}}
\newcommand{\LLL}{\mathcal{L}}
\newcommand{\OOO}{\mathcal{O}}

\newcommand{\T}{\mathrm{T}}
\newcommand{\eqqcolon}{=\mathrel{\mathop:}}
\newcommand{\coloneqq}{\mathrel{\mathop:}=}

\newcounter{blatt}
\newtheoremstyle{jochen}{}{}{}{}{\bfseries}{}{\newline}{}
\theoremstyle{jochen}
\newtheorem{aufg}{Aufgabe}[blatt]
\newtheorem{loes}{L\"osungsskizze}[blatt]
\newtheorem{ex}{Exercise}[blatt]
\newtheorem{sol}{Solution}[blatt]



\newcommand{\tstep}{\Delta t}
\newcommand{\tend}{t_{\mathrm{end}}}
\newcommand{\bigO}{\mathcal{O}}
\newcommand{\quoterat}{\hspace{\fill}\qed}
\newcommand{\zB}{z.\,B.\ }
\newcommand{\iter}{[\nu]}
\newcommand{\transx}{^{\mathrm{T}}}
\newcommand{\partdiff}[1]{\partial_{#1}}
\newcommand{\trace}{\operatorname{trace}}
\newcommand{\radicand}{\mathcal{R}}
%%%%%%%%%%%%%%%%%%%%%%%%%%%%%%%%%%%%%%%%%%%%%%%%%%%%%%%%%%%%%%%%%%%%%%%%%%%%%%%%


\setcounter{blatt}{2}

\begin{document}
%%%%%%%%%%%%%%%%%%%%%%%%%%%%%%%%%%%%%%%%%%%%%%%%%%%%%%%%%%%%%%%%%%%%%%%%%%%%%%%%
% Kopf fuer Uebungszettel                                                      %
% Jochen Siehr                                                                 %
% 2010-10-27                                                                   %
% last-change 2012-10-11                                                       %
%%%%%%%%%%%%%%%%%%%%%%%%%%%%%%%%%%%%%%%%%%%%%%%%%%%%%%%%%%%%%%%%%%%%%%%%%%%%%%%%

\begin{minipage}{0.49\textwidth}
 \begin{flushleft}
  \href{http://www.uni-ulm.de}{Universit\"at Ulm}\\
  \href{http://www.uni-ulm.de/mawi/mawi-numerik.html}{Institut f\"ur Numerische Mathematik}
 \end{flushleft}
\end{minipage}
\begin{minipage}{0.49\textwidth}
 \begin{flushright}
  \href{http://www.lebiedz.de}{Prof. Dr. Dirk Lebiedz}\\
  \href{http://www.lebiedz.de/gruppe/marcfein/index.html}{Marc Fein}\\
  \href{http://www.siehr.net}{Jochen Siehr}
 \end{flushright}
\end{minipage}
\bigskip
\begin{center}
\textbf{
\"Ubungen \theblatt{} 
zur Modellierung und Simulation III
(WS 2012/13)\\
\url{http://www.uni-ulm.de/mawi/mawi-numerik/lehre/wintersemester-20122013/vorlesung-modellierung-und-simulation-3.html}
}
\end{center}
\bigskip
\hrule
\bigskip

%%%%%%%%%%%%%%%%%%%%%%%%%%%%%%%%%%%%%%%%%%%%%%%%%%%%%%%%%%%%%%%%%%%%%%%%%%%%%%%%

%%%%%%%%%%%%%%%%%%%%%%%%%%%%%%%%%%%%%%%%%%%%%%%%%%%%%%%%%%%%%%%%%%%%%%%%%%%%%%%%


\begin{aufg}[Verdünnung einer Lösung]
 Ein Zylinder enthalte 2\,kg Wasser, und es seien 50\,g Saccharose darin
 gelöst. Durch eine Röhre fließt 10\,g Wasser pro Minute in den Zylinder,
 während durch eine andere Röhre 10\,g Wasser pro Minute -- einschließlich eines
 bestimmten Saccharoseanteils -- abfließt. Wie vermindert sich die Masse der
 Saccharose als Funktion der Zeit?
\end{aufg}

%------------------------------------------------------------------------------%
\bigskip%\begin{flushright}b.w.\end{flushright}\pagebreak
%------------------------------------------------------------------------------%

\begin{aufg}[Analytische Lösung von gew.\ Differentialgleichungen]
 Berechnen Sie die analytische Lösung $c(t)$, $v_1(t)$ und $v_2(t)$ der
 folgenden Differentialgleichungen
 für beliebigen Anfangswert.
  \begin{enumerate}[(i)]
  \item $\dot{c}=c^2$
  \item $\dot{v}_1=\frac{v_1}{t}$
  \item $\dot{v}_2=-\frac{v_2}{t}$
 \end{enumerate}
\end{aufg}

%------------------------------------------------------------------------------%
\bigskip%\begin{flushright}b.w.\end{flushright}\pagebreak
%------------------------------------------------------------------------------%

\begin{aufg}[Graphische Analyse]
 Analysieren Sie die folgenden Differentialgleichungen graphisch: Finden Sie
 alle Fixpunkte, klassifizieren Sie deren Stabilität und skizzieren Sie den
 Graph der Lösungen für verschiedene Anfangswerte.
 \begin{enumerate}[(i)]
  \item $\dot{x}=4x^2 - 16$
  \item $\dot{y}=y - y^2$
  \item $\dot{z}=1 + \frac{1}{2}\cos(z)$
 \end{enumerate}
\end{aufg}


%------------------------------------------------------------------------------%
\bigskip%\begin{flushright}b.w.\end{flushright}\pagebreak
%------------------------------------------------------------------------------%

\begin{aufg}[Tumor-Wachstum]
 Das Wachtum eines Tumors kann mit dem Gompertz-Gesetz
 \begin{equation*}
  \dot{N} = - a N \ln(bN)
 \end{equation*}
 besser modelliert werden als mit dem logistischen Wachstumsgesetz nach Verhulst
  \begin{equation*}
  \dot{N} = a(N_{\max}-N) N.
 \end{equation*}
 Dabei ist $N>0$ proportional zur Anzahl der Tumorzellen;
 $a$, $b$ und $N_{\max}$ sind positive Parameter.
 \begin{enumerate}
  \item Skizzieren und vergleichen Sie die Graphen von $N$ für beide Wachstumsgesetze.
  \item Interpretieren Sie $a$, $b$ und $N_{\max}$ biologisch.
  \item Klassifizieren Sie die Fixpunkte der beiden Modelle mit Hilfe der
        linearen Stabilitätsanalyse.
 \end{enumerate}
\end{aufg}


\bigskip
\hrule

%%%%%%%%%%%%%%%%%%%%%%%%%%%%%%%%%%%%%%%%%%%%%%%%%%%%%%%%%%%%%%%%%%%%%%%%%%%%%%%%

\end{document}
