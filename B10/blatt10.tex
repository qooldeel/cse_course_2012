% Uebungsaufgaben zur Vorlesung MoSi 3

\NeedsTeXFormat{LaTeX2e}
\documentclass[11pt,a4paper]{article}
%%%%%%%%%%%%%%%%%%%%%%%%%%%%%%%%%%%%%%%%%%%%%%%%%%%%%%%%%%%%%%%%%%%%%%%%%%%%%%%%
% Header fuer Uebungszettel                                                    %
% Jochen Siehr                                                                 %
% 2010-10-27                                                                   %
% last-change 2010-12-20                                                       %
%%%%%%%%%%%%%%%%%%%%%%%%%%%%%%%%%%%%%%%%%%%%%%%%%%%%%%%%%%%%%%%%%%%%%%%%%%%%%%%%

\usepackage{german}
\usepackage[T1]{fontenc}
\usepackage[utf8]{inputenc}

\usepackage{geometry}
\usepackage{amsmath,amsthm,amssymb}
\usepackage{epsfig}
\usepackage{setspace}
\usepackage{longtable}
\usepackage{enumerate}
\usepackage{eurosym}
\usepackage{siunitx}
\usepackage{pifont}
\usepackage{color}

\usepackage[ % letztes Paket!
 linkbordercolor={1 1 1},
 citebordercolor={1 1 1},
 menubordercolor={1 1 1},
 filebordercolor={1 1 1},
 urlbordercolor={1 1 1},
 pdfborder={0 0 0},
 pdftitle={Uebungsblatt},
 pdfauthor={Jochen Siehr}
 ]{hyperref}
\usepackage{breakurl} %muss danach kommen: AUSKOMMENTIEREN für pdflatex 

\pagestyle{empty}
\geometry{margin=20mm}
\setlength{\parindent}{0em}


\newcommand{\R}{\mathbbm{R}}
\newcommand{\C}{\mathbbm{C}}
\newcommand{\N}{\mathbbm{N}}
\newcommand{\Z}{\mathbbm{Z}}

\newcommand{\CC}{\mathbb{C}}
\newcommand{\NN}{\mathbb{N}}
\newcommand{\QQ}{\mathbb{Q}}
\newcommand{\RR}{\mathbb{R}}
\newcommand{\ZZ}{\mathbb{Z}}

\newcommand{\CCC}{\mathcal{C}}
\newcommand{\NNN}{\mathcal{N}}
\newcommand{\LLL}{\mathcal{L}}
\newcommand{\OOO}{\mathcal{O}}

\newcommand{\T}{\mathrm{T}}
\newcommand{\eqqcolon}{=\mathrel{\mathop:}}
\newcommand{\coloneqq}{\mathrel{\mathop:}=}

\newcounter{blatt}
\newtheoremstyle{jochen}{}{}{}{}{\bfseries}{}{\newline}{}
\theoremstyle{jochen}
\newtheorem{aufg}{Aufgabe}[blatt]
\newtheorem{loes}{L\"osungsskizze}[blatt]
\newtheorem{ex}{Exercise}[blatt]
\newtheorem{sol}{Solution}[blatt]



\newcommand{\tstep}{\Delta t}
\newcommand{\tend}{t_{\mathrm{end}}}
\newcommand{\bigO}{\mathcal{O}}
\newcommand{\quoterat}{\hspace{\fill}\qed}
\newcommand{\zB}{z.\,B.\ }
\newcommand{\iter}{[\nu]}
\newcommand{\transx}{^{\mathrm{T}}}
\newcommand{\partdiff}[1]{\partial_{#1}}
\newcommand{\trace}{\operatorname{trace}}
\newcommand{\radicand}{\mathcal{R}}
%%%%%%%%%%%%%%%%%%%%%%%%%%%%%%%%%%%%%%%%%%%%%%%%%%%%%%%%%%%%%%%%%%%%%%%%%%%%%%%%


\setcounter{blatt}{10}


\begin{document}
%%%%%%%%%%%%%%%%%%%%%%%%%%%%%%%%%%%%%%%%%%%%%%%%%%%%%%%%%%%%%%%%%%%%%%%%%%%%%%%%
% Kopf fuer Uebungszettel                                                      %
% Jochen Siehr                                                                 %
% 2010-10-27                                                                   %
% last-change 2012-10-11                                                       %
%%%%%%%%%%%%%%%%%%%%%%%%%%%%%%%%%%%%%%%%%%%%%%%%%%%%%%%%%%%%%%%%%%%%%%%%%%%%%%%%

\begin{minipage}{0.49\textwidth}
 \begin{flushleft}
  \href{http://www.uni-ulm.de}{Universit\"at Ulm}\\
  \href{http://www.uni-ulm.de/mawi/mawi-numerik.html}{Institut f\"ur Numerische Mathematik}
 \end{flushleft}
\end{minipage}
\begin{minipage}{0.49\textwidth}
 \begin{flushright}
  \href{http://www.lebiedz.de}{Prof. Dr. Dirk Lebiedz}\\
  \href{http://www.lebiedz.de/gruppe/marcfein/index.html}{Marc Fein}\\
  \href{http://www.siehr.net}{Jochen Siehr}
 \end{flushright}
\end{minipage}
\bigskip
\begin{center}
\textbf{
\"Ubungen \theblatt{} 
zur Modellierung und Simulation III
(WS 2012/13)\\
\url{http://www.uni-ulm.de/mawi/mawi-numerik/lehre/wintersemester-20122013/vorlesung-modellierung-und-simulation-3.html}
}
\end{center}
\bigskip
\hrule
\bigskip

%%%%%%%%%%%%%%%%%%%%%%%%%%%%%%%%%%%%%%%%%%%%%%%%%%%%%%%%%%%%%%%%%%%%%%%%%%%%%%%%

%%%%%%%%%%%%%%%%%%%%%%%%%%%%%%%%%%%%%%%%%%%%%%%%%%%%%%%%%%%%%%%%%%%%%%%%%%%%%%%%

\begin{aufg}[Immer noch FitzHugh--Nagumo-Modell]
Das Modell ist gegeben durch
 \begin{align*}
  \dot{u}      =&\ f(u) - v + I_{\rm a}\\
  \dot{v}      =&\ \varepsilon \left(u - \gamma v + \delta\right)\\
  f(u) \coloneqq&\ u\left(a-u\right)\left(u-1\right)
\end{align*}
mit Spannung $u$ und kombinierter Kraft $v$.
Außerdem ist $I_{\rm a}$ eine Stromstärke, die von außen angelegt ist.
Die Parameterwerte sind $\varepsilon = 0.01$, $\gamma = 0.5$,
$\delta = 0$ und $a = -1$.
\medskip

Wir wiederholen Aufgabe~9.2:\\
Ändern Sie im Modell den Parameter $\delta$ auf $\delta = 0.5$,
und wählen Sie $I_{\rm a}=0$.
Setzen Sie die Anfangswerte auf die Werte im Gleichgewicht. Variieren Sie die
angelegte Spannung $I_{\rm a}\in [0,0.04]$ und simulieren Sie das System:
In diesem Zustand heißt das System \emph{erregbar}.
Das Aktionspotential von Nervenzellen kann so simuliert werden.
\end{aufg}

%------------------------------------------------------------------------------%
\bigskip%\begin{flushright}b.w.\end{flushright}\pagebreak
%------------------------------------------------------------------------------%

\begin{aufg}[FitzHugh--Nagumo-Modell: Bifurkationsanalyse]
Wir wollen den Wert von $I_{\rm a}$ bestimmen, bei dem eine Hopf-Bifurkation
auftritt. Dafür nutzen wir das Programm MatCont. Um das Programm zu nutzen,
müssen Sie sich auf bart.mathematik.uni-ulm.de einloggen.

Öffnen Sie eine Shell, und wählen Sie eine Zahl \texttt{XX}$\in[01,\dots,10]$.
Loggen Sie sich auf bart ein, und ändern Sie das Passwort, damit Ihnen niemand
in die Quere kommt.

\begin{verbatim}
> ssh -Y studentXX@bart.mathematik.uni-ulm.de
> passwd
\end{verbatim}

Starten Sie einen vnc-Server und einen vnc-Viewer auf bart, wobei
 $\texttt{YY} \coloneqq \texttt{XX} + 20$.

\begin{verbatim}
> vncstart
> vncviewer :YY
\end{verbatim}

Falls Ihnen gnome nicht zusagt, können Sie in
$\sim$/.vnc/xstartup
auch lxde als graphische Oberfläche wählen. (Dann mit \texttt{vnckill} den vnc
beenden und neu starten.)

Öffnen Sie ein Terminal. Da Sie an einem Mac sitzen, müssen sie noch eine
Umgebungsvariable exportieren, damit MATLAB startet.

\begin{verbatim}
> export LC_ALL="en_US.utf8"
> cd Programmierung/matlab/matcont5p2
> m
\end{verbatim}
In MATLAB starten Sie MatCont mit \texttt{matcont}. Damit sind die
Vorbereitungen erledigt, und die Bifurkationsanalyse kann beginnen.
%------------------------------------------------------------------------------%
\pagebreak
%------------------------------------------------------------------------------%
\begin{enumerate}

\item Geben Sie das Modell in MatCont ein. Dafür wählen Sie im Men\"u
\texttt{Select>System>New}. Das Modell könnte so aussehen:
\begin{figure}[h!]
 \centering
 \includegraphics[width=0.75\textwidth]{model}
\end{figure}

\item Jetzt geben Sie Anfangswerte für die Variablen und Parameter
via \texttt{Type>Initial Point>Point} ein. Au{\ss}erdem k\"onnen Sie die
Integrations-Optionen \"andern. Klicken Sie \emph{nicht} auf
Select Cycle!
\begin{figure}[h!]
 \centering
 \includegraphics[width=0.75\textwidth]{initial}
\end{figure}


\item Wir brauchen noch ein Plot-Fenster: W\"ahlen Sie \texttt{Window>Graphic>2Dplot}.
Auf den Achsen sollten $u$ und $v$ aufgetragen werden. Passen Sie mit
\texttt{Layout>Plotting region} die Plotgrenzen an.

\item Jetzt soll das System nahe ans Gleichgewicht integriert werden. W\"ahlen
Sie \texttt{Compute>Forward}.

\item Wenn Sie vorher \texttt{Window>Numeric} geöffnet hätten, k\"onnten Sie die
numerischen Werte verfolgen. Geben Sie also neue Anfangswerte in den
``Starter'' ein, und integrieren Sie das System erneut mit \texttt{Compute>Forward}.

\item Sollte obiger Schritt nicht funktionieren, geben Sie \texttt{why}
in MATLAB ein.

\item Wir wollen mit einem Fortsetzungs-, Pfadverfolgungs- oder
Homotopieverfahren den Fixpunkt verfolgen, wenn
sich der Parameter $I_{\rm a}$ \"andert. Daf\"ur muss das Gleichgewicht als
Startwert geladen werden. \"Offnen Sie \texttt{Select>Initial Point}. W\"ahlen
Sie einen Wert aus, der nahe am Gleichgewicht liegen sollte. (Anklicken, \texttt{Select}
dr\"ucken.)

\item W\"ahlen Sie \texttt{Type>Initial Point>Equilibrium}. Das Hauptfenster
zeigt an: \texttt{EP\_EP(1)}, d.\,h. ``starte am Gleichgewicht, um ein
Gleichgewicht zu verfolgen''. Sie bekommen zus\"atzlich ein Starter- und ein
Continuer-Fenster.

\item Im Starter aktivieren Sie den Parameter, der variiert werden soll.

\item Im Continuer k\"onnen Sie die maximale Schrittweite etwas kleiner
w\"ahlen, z.\,B.
\begin{figure}[h!]
 \centering
 \includegraphics[width=0.75\textwidth]{starter_continuer}
\end{figure}

\item \"Andern Sie das Plot-Fenster via \texttt{Layout>\dots}, so dass
$I_{\rm a}$ gegen $u$ aufgetragen wird.

\item Nutzen Sie \texttt{Compute>Forward} oder \texttt{Compute>Backward}, um
die Kurve der Nullstellen der rechten Seite der Differentialgleichung f\"ur
verschiedene Werte von $I_{\rm a}$ zu erhalten. Benutzen Sie auch den
Stop- und Resume-Knopf im neu aufgegangenen Fenster. MatCont zeigt eine Hopf-Bifurkation
mit dem Symbol H im Plot an.

\item \"Andern Sie im Numeric-Fenster \"uber \texttt{Window>Layout}, dass auch
die Eigenwerte der Jacobi-Matrix angezeigt werden. Verfolgen Sie, dass an der
Bifurkation tats\"achlich die Eigenwerte die imagin\"are Achse \"uberschreiten.
Bei welchem Wert von $I_{\textrm{a}}^{\textrm{H}}$ tritt die Bifurkation also
wirklich auf?
\end{enumerate}
\end{aufg}


\medskip
\hrule

%%%%%%%%%%%%%%%%%%%%%%%%%%%%%%%%%%%%%%%%%%%%%%%%%%%%%%%%%%%%%%%%%%%%%%%%%%%%%%%%

\end{document}
